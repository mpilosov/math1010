\documentclass[12pt]{article}
\usepackage{epsfig}
\usepackage{amsmath}
\usepackage{amssymb}
\usepackage{graphicx}
\usepackage{float}
\usepackage{epstopdf}

\setlength{\textwidth}{6.5in}
\setlength{\oddsidemargin}{0in}
\setlength{\evensidemargin}{0in}
\setlength{\textheight}{9.5in}
\setlength{\voffset}{-1in}
\setlength{\headsep}{0.25in}
\setlength{\parindent}{0in}


% a few handy macros

\newcommand{\goto}{\rightarrow}
\newcommand{\bigo}{{\mathcal O}}
\newcommand{\half}{\frac{1}{2}}
%\newcommand\implies{\quad\Longrightarrow\quad}
\newcommand\reals{{{\rm l} \kern -.15em {\rm R} }}
\newcommand\complex{{\raisebox{.043ex}{\rule{0.07em}{1.56ex}} \hskip -.35em {\rm C}}}


% macros for matrices/vectors:

% matrix environment for vectors or matrices where elements are centered
\newenvironment{mat}{\left[\begin{array}{ccccccccccccccc}}{\end{array}\right]}
\newcommand\bcm{\begin{mat}}
\newcommand\ecm{\end{mat}}

% matrix environment for vectors or matrices where elements are right justifvied
\newenvironment{rmat}{\left[\begin{array}{rrrrrrrrrrrrr}}{\end{array}\right]}
\newcommand\brm{\begin{rmat}}
\newcommand\erm{\end{rmat}}

% for left brace and a set of choices
\newenvironment{choices}{\left\{ \begin{array}{ll}}{\end{array}\right.}
\newcommand\when{&\text{if~}}
\newcommand\otherwise{&\text{otherwise}}
% sample usage:
%  \delta_{ij} = \begin{choices} 1 \when i=j, \\ 0 \otherwise \end{choices}


% for labeling and referencing equations:
\newcommand{\eql}{\begin{equation}\label}
\newcommand{\eqn}[1]{(\ref{#1})}
% can then do
%  \eql{eqnlabel}
%  ...
%  \end{equation}
% and refer to it as equation \eqn{eqnlabel}.  


% some useful macros for finite difference methods:
\newcommand\unp{U^{n+1}}
\newcommand\unm{U^{n-1}}

% for chemical reactions:
\newcommand{\react}[1]{\stackrel{K_{#1}}{\rightarrow}}
\newcommand{\reactb}[2]{\stackrel{K_{#1}}{~\stackrel{\rightleftharpoons}
   {\scriptstyle K_{#2}}}~}


\usepackage{etoolbox}
\newtoggle{sol}
%\toggletrue{sol}
\togglefalse{sol}
\newtheorem{theorem}{Theorem}
\newtheorem{formula}[theorem]{Formula}

%%%%%%%%%%%%%%%%%%%%%%%%%%%%%%%%%%%%%%%%%%%%%%%%%%%%%%%%%
\begin{document}
\title{\bf Math 1010 \\ Chapter 4B\iftoggle{sol}{ Key}{} }
\date{\today}
\maketitle


%
% a few handy macros

\newcommand{\goto}{\rightarrow}
\newcommand{\bigo}{{\mathcal O}}
\newcommand{\half}{\frac{1}{2}}
%\newcommand\implies{\quad\Longrightarrow\quad}
\newcommand\reals{{{\rm l} \kern -.15em {\rm R} }}
\newcommand\complex{{\raisebox{.043ex}{\rule{0.07em}{1.56ex}} \hskip -.35em {\rm C}}}


% macros for matrices/vectors:

% matrix environment for vectors or matrices where elements are centered
\newenvironment{mat}{\left[\begin{array}{ccccccccccccccc}}{\end{array}\right]}
\newcommand\bcm{\begin{mat}}
\newcommand\ecm{\end{mat}}

% matrix environment for vectors or matrices where elements are right justifvied
\newenvironment{rmat}{\left[\begin{array}{rrrrrrrrrrrrr}}{\end{array}\right]}
\newcommand\brm{\begin{rmat}}
\newcommand\erm{\end{rmat}}

% for left brace and a set of choices
\newenvironment{choices}{\left\{ \begin{array}{ll}}{\end{array}\right.}
\newcommand\when{&\text{if~}}
\newcommand\otherwise{&\text{otherwise}}
% sample usage:
%  \delta_{ij} = \begin{choices} 1 \when i=j, \\ 0 \otherwise \end{choices}


% for labeling and referencing equations:
\newcommand{\eql}{\begin{equation}\label}
\newcommand{\eqn}[1]{(\ref{#1})}
% can then do
%  \eql{eqnlabel}
%  ...
%  \end{equation}
% and refer to it as equation \eqn{eqnlabel}.  


% some useful macros for finite difference methods:
\newcommand\unp{U^{n+1}}
\newcommand\unm{U^{n-1}}

% for chemical reactions:
\newcommand{\react}[1]{\stackrel{K_{#1}}{\rightarrow}}
\newcommand{\reactb}[2]{\stackrel{K_{#1}}{~\stackrel{\rightleftharpoons}
   {\scriptstyle K_{#2}}}~}

  % input some useful macros
%\input{exermacros.tex}       % more macros for exercise formatting


% For exercises,
% set enumerate to give parts a, b, c, ...  rather than numbers 1, 2, 3...
\renewcommand{\theenumi}{\alph{enumi}}
\renewcommand{\labelenumi}{(\theenumi)}

%%%%%%%%%%%%%%%%%%%%%%%%%%%%%%%%%%%%%%%%%%%%%%%%%%
%%%%%%%%%%%%%%%%%%%%%%%%%%%%%%%%%%%%%%%%%%%%%%%%%%

\begin{itemize}
	\item The {\bf principal} in financial formulas is the \emph{balance} on which interest is paid.
	\item {\bf Simple interest} is interest paid \emph{only} on the original principal, and not on top of any interest added at later dates.
	\item {\bf Compounded interest} is interest paid \emph{both} on the original principal \emph{and} on all interest that has been added to the original investment.\item The {\bf Annual Percentage Yield (APY)}\---also called the \emph{effective yield} or simply the \emph{yield}\---is the actual percentage by which a balance increases in one year.  It is equal to the APR if the interest is compounded yearly.  It is {\it greater than} the APR if interest is compounded more than once a year.
\end{itemize}

\begin{formula}[Simple Interest]
	Every time period you receive the same interest payment.
	\begin{equation}\label{eq:simple}
		\text{Interest Payment per Time Period} = P\times(APR),
	\end{equation}
	where
	\begin{itemize}
		\item $P$ is the starting principal
		\item APR is the Annual Percentage Rate  (as a decimal)
	\end{itemize}
\end{formula}

\begin{formula}[Compound Interest]
	\begin{equation}\label{eq:compound}
		A = P\Big(1 + \frac{APR}{n} \Big)^{nY},
	\end{equation}
	where
	\begin{itemize}
		\item $A$ is the accumulated balance after $Y$ years
		\item $P$ is the starting principal
		\item APR is the Annual Percentage Rate (as a decimal)
		\item $n$ is the number of compounding periods per year
		\item and $Y$ is the number of years
	\end{itemize}
\end{formula}

\pagebreak

\begin{formula}[Continuous Compound Interest]
	\begin{equation}\label{eq:continuous}
		A = Pe^{(APR\times Y)}
	\end{equation}
	where
	\begin{itemize}
		\item $A$ is the accumulated balance after $Y$ years
		\item $P$ is the starting principal
		\item APR is the Annual Percentage Rate (as a decimal)
		\item $e$ is a special irrational number with a value of $e\approx 2.71828$
		\item and $Y$ is the number of years
	\end{itemize}
\end{formula}


%%%%%%%%%%%%%%%%%%%%%%%%%%%%%%%%%%%%%%%%%%%%%%%%%%
%%%%%%%%%%%%%%%%%%%%%%%%%%%%%%%%%%%%%%%%%%%%%%%%%%
\leftline{\bf \large Exercise 1 : Simple Interest}
Bonds typically pay simple interest.  Suppose you invest $\$1000$ in a bond that pays simple interest of $5\%$ per year.

\begin{enumerate}
\item How much interest will you receive each year?

\iftoggle{sol}{
{\it Solution:

We use Eq.~\eqref{eq:simple} to find the interest payment per year:
\begin{eqnarray*}
	\text{Interest Payment per Year} &=& P(APR) \\
	&=& \$1000(0.05) \\
	&=& \$50
\end{eqnarray*}
}
}{\vspace{1in}}
%%%%%%%%%%%%%%%%%%%%
\item How much total interest will you receive in 5 years?

\iftoggle{sol}{
{\it Solution:

From part (a), each year we receive a payment of $\$50$ from the interest earned on the bond.  So the total interest earned in five years is
\begin{eqnarray*}
	5 \times \$50 = \$250
\end{eqnarray*}
}
}{\vspace{1in}}
%%%%%%%%%%%%%%%%%%%%
\item If the bond paid compound interest, would you receive more or less total interest? Explain.

\iftoggle{sol}{
{\it Solution:
We would receive more interest because compound interest ALWAYS yields more than simple interest.  However, without knowing the time period of the compound interest, we cannot say exactly how much more.
}
}{\vspace{1in}}

\end{enumerate}


\pagebreak

%%%%%%%%%%%%%%%%%%%%%%%%%%%%%%%%%%%%%%%%%%%%%%%%%%
%%%%%%%%%%%%%%%%%%%%%%%%%%%%%%%%%%%%%%%%%%%%%%%%%%
\leftline{\bf \large Exercise 2 : Compound Interest}
Suppose you deposit $\$1000$ into an account with an APR of $5\%$
\begin{enumerate}
\item Suppose it compounds interest monthly. How much money will you have after 5 years?

\iftoggle{sol}{
{\it Solution:
We use Eq.~\eqref{eq:compound} to find the accumulated balance after 5 years:
\begin{eqnarray*}
	A &=& P\Big(1 + \frac{APR}{n} \Big)^{nY} \\
	&=& 1000\Big(1 + \frac{0.05}{12}\Big)^{(12\times 5)} \\
	&=& \$1283.36
\end{eqnarray*}
}
}{\vspace{1in}}
%%%%%%%%%%%%%%%%%%%%
\item What if the account compounded interest daily?  Now how much money will you have after 5 years?  Will it be more or less than compounded monthly?

\iftoggle{sol}{
{\it Solution:

We use Eq.~\ref{eq:compound} to find the accumulated balance after 5 years:
\begin{eqnarray*}
	A &=& P\Big(1 + \frac{APR}{n} \Big)^{nY} \\
	&=& 1000\Big(1 + \frac{0.05}{365}\Big)^{(365\times 5)} \\
	&=& \$1284.00
\end{eqnarray*}
}
}{\vspace{1in}}
%%%%%%%%%%%%%%%%%%%%
\item What if the account compounded interest hourly?  Now how much money will you have after 5 years?  Will it be more or less than if it compounded daily?

\iftoggle{sol}{
{\it Solution:}
\textbf{More frequent compounding means more money.}
If we compounded hourly, then we have:
$$
1 \text{ year} \times \frac{365\text{ days}}{1\text{ year}} \times \frac{24 \text{ hours}}{1\text{ day}} = 8,760 \text{ hours}
$$
\begin{eqnarray*}
	A &=& P\Big(1 + \frac{APR}{n} \Big)^{nY} \\
	&=& 1000\Big(1 + \frac{0.05}{8760}\Big)^{(8760\times  5)} \\
	&=& \$1284.02
\end{eqnarray*}
}{\vspace{1in}}


\end{enumerate} 

%%%%%%%%%%%%%%%%%%%%%%%%%%%%%%%%%%%%%%%%%%%%%%%%%%
%%%%%%%%%%%%%%%%%%%%%%%%%%%%%%%%%%%%%%%%%%%%%%%%%%
\leftline{\bf \large Exercise 3 : Continuous Compound Interest}
Suppose you deposit $\$1000$ into an account with an APR of $5\%$ and \emph{continuous} compounding interest.
How much money will you have after 5 years?

\iftoggle{sol}{
{\it Solution:

We use Eq.~\eqref{eq:continuous} to find the accumulated balance after 5 years:
\begin{eqnarray*}
	A &=& Pe^{(APR\times Y)} \\
	&=& 1000e^{(0.05\times 5)} \\
	&=& \$1284.03
\end{eqnarray*}
}
}{\vspace{1in}}

This represents the upper limit on how much your money could grow at this interest rate over this time period. 



\end{document}
