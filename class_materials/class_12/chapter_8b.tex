\documentclass[12pt]{article}
\usepackage{epsfig}
\usepackage{amsmath}
\usepackage{amssymb}
\usepackage{graphicx}
\usepackage{float}
\usepackage{epstopdf}

\setlength{\textwidth}{6.5in}
\setlength{\oddsidemargin}{0in}
\setlength{\evensidemargin}{0in}
\setlength{\textheight}{9.5in}
\setlength{\voffset}{-1in}
\setlength{\headsep}{0.25in}
\setlength{\parindent}{0in}

\input{../macros}
\usepackage{etoolbox}

\newtoggle{sol}
%\toggletrue{sol}

\togglefalse{sol}
\newtheorem{theorem}{Theorem}
\newtheorem{formula}[theorem]{Formula}

%%%%%%%%%%%%%%%%%%%%%%%%%%%%%%%%%%%%%%%%%%%%%%%%%%%%%%%%%
\begin{document}
\title{\bf Math 1010 \\ Chapter 8B\iftoggle{sol}{ Key}{} }
\date{Feb 26 2018}
\maketitle

%\input{./latex/macros.tex}  % input some useful macros
%\input{exermacros.tex}       % more macros for exercise formatting


% For exercises,
% set enumerate to give parts a, b, c, ...  rather than numbers 1, 2, 3...
\renewcommand{\theenumi}{\alph{enumi}}
\renewcommand{\labelenumi}{(\theenumi)}

%%%%%%%%%%%%%%%%%%%%%%%%%%%%%%%%%%%%%%%%%%%%%%%%%%
%%%%%%%%%%%%%%%%%%%%%%%%%%%%%%%%%%%%%%%%%%%%%%%%%%
%%%%%%%%%%%%%%%%%%%%%%%%%%%%%%%%%%%%%%%%%%%%%%%%%%
%%%%%%%%%%%%%%%%%%%%%%%%%%%%%%%%%%%%%%%%%%%%%%%%%%
{\large \bf Doubling and Halving Times}
\begin{itemize}
	\item The time required for each doubling in exponential growth is call the {\bf doubling time}
	\item The time required for each halving (reduction in the quantity by $50\%$) in exponential growth is called the {\bf halving time}.
\end{itemize}

\begin{formula}[Doubling Time]
After a time $t$, an exponentially growing quantity with a doubling time of $T_{double}$ increases in size by a factor of $$2^{t / T_{double}}.$$  The new value of the quantity is then given by
\begin{equation}\label{eq:doubling_time}
	\text{New Value} = \text{Initial Value} \times 2^{t / T_{double}}
\end{equation}

\end{formula}

\begin{formula}[APPROXIMATE Doubling Time]
For a quantity growing exponentially at a rate of $P\%$ per time period, the doubling time is APPROXIMATELY
\begin{equation}\label{eq:approximate_doubling}
	T_{double} \approx \frac{70}{P}.
\end{equation}

This approximation works best for small growth rates and {\bf does not work well for growth rates over about $15\%$} (sounds like a good True/False exam question $\hdots$).

\end{formula}

\begin{formula}[Half-Life]
After a time $t$, an exponentially decaying quantity with a {\bf half-life time of $T_{half}$} decreases in size by a factor of $$\Big(\frac{1}{2}\Big)^{t / T_{half}}.$$ The new value of the decaying quantity is related to its initial value (at $t=0$) by
\begin{equation}\label{eq:half_life}
	\text{New value} = \text{initial value} \times \Big(\frac{1}{2}\Big)^{t / T_{half}}
\end{equation}


\end{formula}

\begin{formula}[APPROXIMATE Half-Life Formula]
For a quantity decaying exponentially at a rate of $P\%$ per time period, the half-life is APPROXIMATELY
\begin{equation}\label{eq:approximate_half_life}
	T_{half} \approx \frac{70}{P}.
\end{equation}

This approximation works best for small growth rates and {\bf does not work well for decay rates over about $15\%$} (also a good True/False exam question ...).

\end{formula}


------------------------------------------------------------------------------------------------------------------

%%%%%%%%%%%%%%%%%%%%%%%%%%%%%%%%%%%%%%%%%%%%%%%%%%
%%%%%%%%%%%%%%%%%%%%%%%%%%%%%%%%%%%%%%%%%%%%%%%%%%
\leftline{\bf \large Exercise 1 : Doubling Time}
Compound interest produces exponential growth because an interest-bearing account grows by the same percentage each year \emph{(yet another good True/False exam question)}. Suppose your bank account has a doubling time of 13 years.

\begin{enumerate}
\item By what factor does your balance increase in 50 years?
\iftoggle{sol}{
{\it Solution:
The doubling time is $T_{double} = 13$ years, so after $t=50$ years, we use Eq.~\eqref{eq:doubling_time} to find the factor by which your balance increases,
\begin{eqnarray*}
	2^{t / T_{double}} &=& 2^{50 / 13} \\
	&=& 2^{3.8462} \\
	&\approx & 14.382.
\end{eqnarray*}
}}{\vspace{1in}}



\item If you started with a balance of $\$1,000$, how much will it have grown to after 50 years?
\iftoggle{sol}{
{\it Solution:
Again we refer to Eq.~\eqref{eq:doubling_time}, where this time we use $\$1,000$ as the {\bf initial value} of the account,
\begin{eqnarray*}
	\text{New Value} &=& \text{initial value} \times 2^{t / T_{double}} \\
	&=& (1000)\times(2^{50/13}) \\
	&=& \$14,382
\end{eqnarray*}

}
}{\vspace{1in}}


\end{enumerate}

%%%%%%%%%%%%%%%%%%%%%%%%%%%%%%%%%%%%%%%%%%%%%%%%%%
%%%%%%%%%%%%%%%%%%%%%%%%%%%%%%%%%%%%%%%%%%%%%%%%%%
\leftline{\bf \large Exercise 2 : APPROXIMATE Doubling Time}
The world population doubled in the 40 years from 1960 to 2000.

\begin{enumerate}
\item What was the average percentage growth rate during this time period?
\iftoggle{sol}{
{\it Solution:
We use Eq.~\eqref{eq:approximate_doubling} to find $P$,
\begin{eqnarray*}
	T_{double} &=& \frac{70}{P} \\
	40 &=& \frac{70}{P} \\
	40P &=& 70 \\
	P &=& 1.75.
\end{eqnarray*}
So the average population growth rate between 1960 and 2000 was about $P=1.75\%$ per year.

}
}{\vspace{1in}}

\item Suppose the growth rate was $P=5\%$.  How long would it have taken for the world population to double?
\iftoggle{sol}{
{\it Solution:
We use Eq.~\eqref{eq:approximate_doubling} to find $T_{double}$,
\begin{eqnarray*}
	T_{double} &=& \frac{70}{P} \\
	&=& \frac{70}{5} \\
	&=& 14
\end{eqnarray*}
So it would have taken just 14 years for the population to double, this would be 1974.

}
}{\vspace{1in}}

\end{enumerate}


%%%%%%%%%%%%%%%%%%%%%%%%%%%%%%%%%%%%%%%%%%%%%%%%%%
%%%%%%%%%%%%%%%%%%%%%%%%%%%%%%%%%%%%%%%%%%%%%%%%%%
\leftline{\bf \large Exercise 3 : Half-Life}
Radioactive cabon-14 has a half-life of about 5700 years.  It collects in organisms only while they are alive.  Once they are dead, it only decays.

\begin{enumerate}
\item What fraction of the carbon-14 is an animals bone still remains 1,000 years after the animal has died?
\iftoggle{sol}{
{\it Solution:
We use Eq.~\eqref{eq:half_life} to determine this fraction,
\begin{eqnarray*}
	\Big(\frac{1}{2}\Big)^{t / T_{half}} &=& \Big(\frac{1}{2}\Big)^{1000 / 5700} \\
	&=& 0.885
\end{eqnarray*}
So $88.5\%$ of the carbon-14 remains in the animal after 1,000 years.

}
}{\vspace{1in}}

\item After 10,000 years, what fraction of the carbon-14 remains in the animal?
\iftoggle{sol}{
{\it Solution:
We use Eq.~\eqref{eq:half_life} to determine this fraction,
\begin{eqnarray*}
	\Big(\frac{1}{2}\Big)^{t / T_{half}} &=& \Big(\frac{1}{2}\Big)^{10,000 / 5700} \\
	&=& 0.885
\end{eqnarray*}
So $29.6\%$ of the carbon-14 remains in the animal after 10,000 years.

}
}{\vspace{1in}
\item After 50,000 years, what fraction of the carbon-14 remains in the animal?
\iftoggle{sol}{
{\it Solution:
We use Eq.~\eqref{eq:half_life} to determine this fraction,
\begin{eqnarray*}
	\Big(\frac{1}{2}\Big)^{t / T_{half}} &=& \Big(\frac{1}{2}\Big)^{50,000 / 5700} \\
	&=& 0.002287631157229656
\end{eqnarray*}
So $0.229\%$ of the carbon-14 remains in the animal after 50,000 years.

}
}{\vspace{1in}}

\end{enumerate}

\pagebreak

%%%%%%%%%%%%%%%%%%%%%%%%%%%%%%%%%%%%%%%%%%%%%%%%%%
%%%%%%%%%%%%%%%%%%%%%%%%%%%%%%%%%%%%%%%%%%%%%%%%%%
\leftline{\bf \large Exercise 4 : APPROXIMATE Half-Life}
Suppose that inflation causes the value of the Russian currency (ruble) to fall at a rate of $10\%$ per year (relative to the US dollar).

\begin{enumerate}
\item At this rate, approximately how long does it take for the ruble to lose half of its value?

{\it Solution:
We use the APPROXIMATE half-life formula Eq.~\eqref{eq:approximate_half_life},
\begin{eqnarray*}
	T_{half} &=& \frac{70}{P} \\
	&=& \frac{70}{10} \\
	&=& 7
\end{eqnarray*}
So it will take about 7 years for the the ruble to lose half of its value.

The \emph{true} answer is actually closer to 6.5 years. 
}
}{\vspace{1in}}
\vspace{1.5in}

\item How long will it take to lose $\frac{3}{4}$ of its value? \emph{Hint: same procedure as (a) but with a different ``starting'' value. Solve for the time it would take to halve its value again (half of half means a fourth of the original value is left, so it lost $\frac{3}{4}$ of its value), and add it to the time you found from (a). }
\iftoggle{sol}{
{\it Solution:
We use the APPROXIMATE half-life formula again, but the starting amount is now  Eq.~\eqref{eq:approximate_half_life},
\begin{eqnarray*}
	T_{half} &=& \frac{70}{P} \\
	&=& \frac{70}{10} \\
	&=& 7
\end{eqnarray*}
So it will take about 7+7=14 years for the the ruble to lose half of its value.

The \emph{true} answer is actually closer to 13 years. The approximation works best for shorter time periods.
}
}{\vspace{2in}}
\end{enumerate}


%%%%%%%%%%%%%%%%%%%%%%%%%%%%%%%%%%%%%%%%%%%%%%%%%%
%%%%%%%%%%%%%%%%%%%%%%%%%%%%%%%%%%%%%%%%%%%%%%%%%%
%%%%%%%%%%%%%%%%%%%%%%%%%%%%%%%%%%%%%%%%%%%%%%%%%%
%%%%%%%%%%%%%%%%%%%%%%%%%%%%%%%%%%%%%%%%%%%%%%%%%%
\end{document}
