
\documentclass[12pt]{article}
\usepackage{epsfig}
\usepackage{amsmath}
\usepackage{amssymb}
\usepackage{graphicx}
\usepackage{float}
\usepackage{epstopdf}

\setlength{\textwidth}{6.2in}
\setlength{\oddsidemargin}{0.3in}
\setlength{\evensidemargin}{0in}
\setlength{\textheight}{8.7in}
\setlength{\voffset}{-.7in}
\setlength{\headsep}{26pt}
\setlength{\parindent}{10pt}
%%%%%%%%%%%%%%%%%%%%%%%%%%%%%%%%%%%%%%%%%%%%%%%%%%%%%%%%%
\begin{document}
\title{\bf Math 1010 \\ Chapter 1C}
\date{Jan 17, 2018}
\maketitle

%\vskip .3in

\input{../macros}
%\input{./latex/macros.tex}  % input some useful macros
%\input{exermacros.tex}       % more macros for exercise formatting


% For exercises,
% set enumerate to give parts a, b, c, ...  rather than numbers 1, 2, 3...
\renewcommand{\theenumi}{\alph{enumi}}
\renewcommand{\labelenumi}{(\theenumi)}

%%%%%%%%%%%%%%%%%%%%%%%%%%%%%%%%%%%%%%%%%%%%%%%%%%
%%%%%%%%%%%%%%%%%%%%%%%%%%%%%%%%%%%%%%%%%%%%%%%%%%
%%%%%%%%%%%%%%%%%%%%%%%%%%%%%%%%%%%%%%%%%%%%%%%%%%
%%%%%%%%%%%%%%%%%%%%%%%%%%%%%%%%%%%%%%%%%%%%%%%%%%
\begin{itemize}
	\item The {\bf real numbers} consist of the entire number line.  They contain both the rational and irrational numbers.
	\item The {\bf rational numbers} are the set of numbers that can be expressed in the form
	\begin{equation*}
		\frac{x}{y}, \textsf{where $x$ and $y$ are integers and $y\neq 0$}
	\end{equation*}
	\item The {\bf irrational numbers} are the set of number that {\it cannot} be expressed in the form $x/y$, where $x$ and $y$ are integers.
	\item The {\bf integers} are the set of numbers
	\begin{equation*}
		\{\dots, -3, -2, -1, 0, 1, 2, 3, \dots\}
	\end{equation*}
	\item The {\bf whole numbers} are the set of {\it non-negative} integers
	\begin{equation*}
		\{0, 1, 2, 3, \dots\}
	\end{equation*}
	\item The {\bf natural numbers} are just the set of {\it positive} integers.  So they are the whole numbers, \emph{ but without $0$.}
	\begin{equation*}
		\{1, 2, 3, \dots\}
	\end{equation*} 
	\item We use brackets/braces $\lbrace$ and $\rbrace$ to represent collections of numbers, items, etc. They denote \emph{sets.}
	\item We use intervals $($ and $)$ to represent \emph{intervals.}
\end{itemize}

\pagebreak


%%%%%%%%%%%%%%%%%%%%%%%%%%%%%%%%%%%%%%%%%%%%%%%%%%
%%%%%%%%%%%%%%%%%%%%%%%%%%%%%%%%%%%%%%%%%%%%%%%%%%
\leftline{\bf \large Exercise 1 : Number Sets}
In the space below, try to give some examples of each number set. \\
\emph{If you get stuck, ask me or your group members for help.}

\begin{enumerate}
	\item Examples of real numbers
	\vskip 30pt
	\item Examples of rational numbers
	\vskip 30pt
	\item Examples of irrational numbers
	\vskip 30pt
	\item Examples of integers
	\vskip 30pt
	\item Examples of whole numbers
	\vskip 30pt
	\item Examples of natural numbers
	\vskip 30pt
\end{enumerate}


\pagebreak
%%%%%%%%%%%%%%%%%%%%%%%%%%%%%%%%%%%%%%%%%%%%%%%%%%
%%%%%%%%%%%%%%%%%%%%%%%%%%%%%%%%%%%%%%%%%%%%%%%%%%
\leftline{\bf \large Exercise 2 : Venn Diagram with the Number Sets}
Now, fill out the Venn diagram using what you know about the real numbers, rational numbers, irrational numbers, integers, whole numbers, and natural numbers.

\vskip 30pt

\includegraphics[scale=.8]{numbers_empty.png}

\pagebreak
%%%%%%%%%%%%%%%%%%%%%%%%%%%%%%%%%%%%%%%%%%%%%%%%%%
%%%%%%%%%%%%%%%%%%%%%%%%%%%%%%%%%%%%%%%%%%%%%%%%%%
\leftline{\bf \large Exercise 3 : Set Notation}
Use braces to write the members of the following sets: \\

\begin{enumerate}
	\item The integers between 0 and 4 (inclusive)
	\vskip 30pt
	\item The even numbers between 2 and 100  (inclusive)
	\vskip 30pt
	\item The integers
	\vskip 30pt
\end{enumerate}

\vspace{0.5in}
%%%%%%%%%%%%%%%%%%%%%%%%%%%%%%%%%%%%%%%%%%%%%%%%%%
%%%%%%%%%%%%%%%%%%%%%%%%%%%%%%%%%%%%%%%%%%%%%%%%%%
\leftline{\bf \large Exercise 4 : More Venn Diagrams}
Draw a Venn diagram with two circles showing the given relationship between the following pairs of categories: \\

\begin{enumerate}
	\item teachers and fish
	\vskip 80pt
	\item fish and tuna fish
	\vskip 80pt
	\item teachers and people who like to fish
	\vskip 80pt
\end{enumerate}









%%%%%%%%%%%%%%%%%%%%%%%%%%%%%%%%%%%%%%%%%%%%%%%%%%
%%%%%%%%%%%%%%%%%%%%%%%%%%%%%%%%%%%%%%%%%%%%%%%%%%
%%%%%%%%%%%%%%%%%%%%%%%%%%%%%%%%%%%%%%%%%%%%%%%%%%
%%%%%%%%%%%%%%%%%%%%%%%%%%%%%%%%%%%%%%%%%%%%%%%%%%
\end{document}
