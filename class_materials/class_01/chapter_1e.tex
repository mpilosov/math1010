\documentclass[12pt]{article}
\usepackage{epsfig}
\usepackage{amsmath}
\usepackage{amssymb}
\usepackage{graphicx}
\usepackage{float}
\usepackage{epstopdf}

\setlength{\textwidth}{6.2in}
\setlength{\oddsidemargin}{0.3in}
\setlength{\evensidemargin}{0in}
\setlength{\textheight}{8.7in}
\setlength{\voffset}{-.7in}
\setlength{\headsep}{26pt}
\setlength{\parindent}{10pt}
%%%%%%%%%%%%%%%%%%%%%%%%%%%%%%%%%%%%%%%%%%%%%%%%%%%%%%%%%
\begin{document}
\title{\bf Math 1010 \\ Chapter 1E}
\date{Jan 17, 2018}
\maketitle

%\vskip .3in

\input{../macros}
%\input{./latex/macros.tex}  % input some useful macros
%\input{exermacros.tex}       % more macros for exercise formatting


% For exercises,
% set enumerate to give parts a, b, c, ...  rather than numbers 1, 2, 3...
\renewcommand{\theenumi}{\alph{enumi}}
\renewcommand{\labelenumi}{(\theenumi)}

%%%%%%%%%%%%%%%%%%%%%%%%%%%%%%%%%%%%%%%%%%%%%%%%%%
%%%%%%%%%%%%%%%%%%%%%%%%%%%%%%%%%%%%%%%%%%%%%%%%%%
%%%%%%%%%%%%%%%%%%%%%%%%%%%%%%%%%%%%%%%%%%%%%%%%%%
%%%%%%%%%%%%%%%%%%%%%%%%%%%%%%%%%%%%%%%%%%%%%%%%%%
The following are a list of skills you should keep in mind and exercise when approaching a new or unfamiliar problem:
\begin{itemize}
	\item Read Carefully
	\item Look for Hidden Assumptions
	\item Identify the Real Issue
	\item Understand All the Options
	\item Watch For Fine Print and Misleading Information
	\item Are Other Conclusions Possible?
\end{itemize}

\pagebreak


%%%%%%%%%%%%%%%%%%%%%%%%%%%%%%%%%%%%%%%%%%%%%%%%%%
%%%%%%%%%%%%%%%%%%%%%%%%%%%%%%%%%%%%%%%%%%%%%%%%%%
\leftline{\bf \large Exercise 1:  Warm-up Problems}

\begin{enumerate}
	\item Jose had 6 bagels and ate all but 4 of them. How many bagels were left? Explain.
	\vspace{1in}
	\item 	Paris Hilton’s rooster laid an egg in Britney Spears’ yard. Who owns the egg? Explain.
	\vspace{1in}
	\item Find two hidden assumptions in the following statement: \emph{Buying a house today makes good sense. The rent money you save can be put into a long-term investment.}
	\vspace{1in}
	\item Suppose you go to a conference attended by 20 Canadians and 20 Norwegians. How many people must you meet to be certain that you have met one Norwegian and one Canadian? Explain.
	\vspace{1in}
\end{enumerate}


\pagebreak
%%%%%%%%%%%%%%%%%%%%%%%%%%%%%%%%%%%%%%%%%%%%%%%%%%
%%%%%%%%%%%%%%%%%%%%%%%%%%%%%%%%%%%%%%%%%%%%%%%%%%
\leftline{\bf \large Exercise 2 : Harry Potter Reasoning}
Solve this classic riddle in your small groups. The image is labeled with letters representing the possibilities for each bottle so you can cross out options as you reason through the text. (Hint: solve this problem by a process of elimination.)  \\

\noindent \emph{Danger lies before you, while safety lies behind, \\
Two of us will help you, whichever you would find,\\
One among us seven will let you move ahead,\\
Another will transport the drinker back instead,\\
Two among our number hold only nettle wine,\\
Three of us are killers, waiting hidden in line.\\
Choose, unless you wish to stay here for evermore,\\
To help you in your choice, we give you these clues four:\\}

\noindent \emph{First, however slyly the poison tries to hide\\
You will always find some on nettle wine’s left side;\\
Second, different are those who stand at either end,\\
But if you would move onwards, neither is your friend;\\
Third, as you see clearly, all are different size,\\
Neither dwarf nor giant holds death in their insides;\\
Fourth, the second left and the second on the right\\
Are twins once you taste them, though different at first sight.\\}

- J.K. Rowling, 1998 \emph{Harry Potter and the Philosopher's Stone}, pp. 285
\vspace{0.25in}

\includegraphics[scale=.4]{harrypotter.png}


%%%%%%%%%%%%%%%%%%%%%%%%%%%%%%%%%%%%%%%%%%%%%%%%%%
%%%%%%%%%%%%%%%%%%%%%%%%%%%%%%%%%%%%%%%%%%%%%%%%%%
%%%%%%%%%%%%%%%%%%%%%%%%%%%%%%%%%%%%%%%%%%%%%%%%%%
%%%%%%%%%%%%%%%%%%%%%%%%%%%%%%%%%%%%%%%%%%%%%%%%%%
\end{document}
