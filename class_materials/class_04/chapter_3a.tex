\documentclass[12pt]{article}
\usepackage{epsfig}
\usepackage{amsmath}
\usepackage{amssymb}
\usepackage{graphicx}
\usepackage{float}
\usepackage{epstopdf}

\setlength{\textwidth}{6.5in}
\setlength{\oddsidemargin}{0in}
\setlength{\evensidemargin}{0in}
\setlength{\textheight}{9.5in}
\setlength{\voffset}{-1in}
\setlength{\headsep}{0.25in}
\setlength{\parindent}{0in}

%%%%%%%%%%%%%%%%%%%%%%%%%%%%%%%%%%%%%%%%%%%%%%%%%%%%%%%%%
\begin{document}
\title{\bf Math 1010 \\ Chapter 3A}
\date{Jan 29, 2018}
\maketitle
%\vskip .3in


% a few handy macros

\newcommand{\goto}{\rightarrow}
\newcommand{\bigo}{{\mathcal O}}
\newcommand{\half}{\frac{1}{2}}
%\newcommand\implies{\quad\Longrightarrow\quad}
\newcommand\reals{{{\rm l} \kern -.15em {\rm R} }}
\newcommand\complex{{\raisebox{.043ex}{\rule{0.07em}{1.56ex}} \hskip -.35em {\rm C}}}


% macros for matrices/vectors:

% matrix environment for vectors or matrices where elements are centered
\newenvironment{mat}{\left[\begin{array}{ccccccccccccccc}}{\end{array}\right]}
\newcommand\bcm{\begin{mat}}
\newcommand\ecm{\end{mat}}

% matrix environment for vectors or matrices where elements are right justifvied
\newenvironment{rmat}{\left[\begin{array}{rrrrrrrrrrrrr}}{\end{array}\right]}
\newcommand\brm{\begin{rmat}}
\newcommand\erm{\end{rmat}}

% for left brace and a set of choices
\newenvironment{choices}{\left\{ \begin{array}{ll}}{\end{array}\right.}
\newcommand\when{&\text{if~}}
\newcommand\otherwise{&\text{otherwise}}
% sample usage:
%  \delta_{ij} = \begin{choices} 1 \when i=j, \\ 0 \otherwise \end{choices}


% for labeling and referencing equations:
\newcommand{\eql}{\begin{equation}\label}
\newcommand{\eqn}[1]{(\ref{#1})}
% can then do
%  \eql{eqnlabel}
%  ...
%  \end{equation}
% and refer to it as equation \eqn{eqnlabel}.  


% some useful macros for finite difference methods:
\newcommand\unp{U^{n+1}}
\newcommand\unm{U^{n-1}}

% for chemical reactions:
\newcommand{\react}[1]{\stackrel{K_{#1}}{\rightarrow}}
\newcommand{\reactb}[2]{\stackrel{K_{#1}}{~\stackrel{\rightleftharpoons}
   {\scriptstyle K_{#2}}}~}


%
% a few handy macros

\newcommand{\goto}{\rightarrow}
\newcommand{\bigo}{{\mathcal O}}
\newcommand{\half}{\frac{1}{2}}
%\newcommand\implies{\quad\Longrightarrow\quad}
\newcommand\reals{{{\rm l} \kern -.15em {\rm R} }}
\newcommand\complex{{\raisebox{.043ex}{\rule{0.07em}{1.56ex}} \hskip -.35em {\rm C}}}


% macros for matrices/vectors:

% matrix environment for vectors or matrices where elements are centered
\newenvironment{mat}{\left[\begin{array}{ccccccccccccccc}}{\end{array}\right]}
\newcommand\bcm{\begin{mat}}
\newcommand\ecm{\end{mat}}

% matrix environment for vectors or matrices where elements are right justifvied
\newenvironment{rmat}{\left[\begin{array}{rrrrrrrrrrrrr}}{\end{array}\right]}
\newcommand\brm{\begin{rmat}}
\newcommand\erm{\end{rmat}}

% for left brace and a set of choices
\newenvironment{choices}{\left\{ \begin{array}{ll}}{\end{array}\right.}
\newcommand\when{&\text{if~}}
\newcommand\otherwise{&\text{otherwise}}
% sample usage:
%  \delta_{ij} = \begin{choices} 1 \when i=j, \\ 0 \otherwise \end{choices}


% for labeling and referencing equations:
\newcommand{\eql}{\begin{equation}\label}
\newcommand{\eqn}[1]{(\ref{#1})}
% can then do
%  \eql{eqnlabel}
%  ...
%  \end{equation}
% and refer to it as equation \eqn{eqnlabel}.  


% some useful macros for finite difference methods:
\newcommand\unp{U^{n+1}}
\newcommand\unm{U^{n-1}}

% for chemical reactions:
\newcommand{\react}[1]{\stackrel{K_{#1}}{\rightarrow}}
\newcommand{\reactb}[2]{\stackrel{K_{#1}}{~\stackrel{\rightleftharpoons}
   {\scriptstyle K_{#2}}}~}

  % input some useful macros
%\input{exermacros.tex}       % more macros for exercise formatting


% For exercises,
% set enumerate to give parts a, b, c, ...  rather than numbers 1, 2, 3...
%\renewcommand{\theenumi}{\alph{enumi}}
%\renewcommand{\labelenumi}{(\theenumi)}


%%%%%%%%%%%%%%%%%%%%%%%%%%%%%%%%%%%%%%%%%%%%%%%%%%
%%%%%%%%%%%%%%%%%%%%%%%%%%%%%%%%%%%%%%%%%%%%%%%%%%

\leftline{\bf \large Uses of Percentages }
We have three primary ways that we will be using percentages:
\begin{enumerate}
\item To describe fractions (proportions)\\
\emph{Of the total 13,000 work force, 2.6\% lost their jobs.}
\item To describe change\\
\emph{The stock of a company fell 15\% last week to a value of \$44.25}
\item For (relative) comparisons\\
\emph{The new battery lasts 125\% longer than the standard one but costs 200\% more.}
\end{enumerate}

%%%%%%%%%%%%%%%%%%%%%%%%%%%%%%%%%%%%%%%%%%%%%%%%%%
%%%%%%%%%%%%%%%%%%%%%%%%%%%%%%%%%%%%%%%%%%%%%%%%%%

\leftline{\bf \large Examples }
A Gallup poll for August 18-24 2017 found that 43\% of 3548 people surveyed approved of the way that the President is doing his job. How many said that they approved of the way the President is doing his job?\\

\emph{of} indicates multiplication. 43\% of 3548 respondents is 
\[
43\% \times 3548 = 0.42 \times 3548 = 1525.64 \approx 1526 \text{ people}
\]

The same Gallup poll found that 51\% of the same people surveyed disapproved of the job the President was doing. How many said they disapproved?
\vspace{0.5in}

%%%%%%%%%%%%%%%%%%%%%%%%%%%%%%%%%%%%%%%%%%%%%%%%%%
%%%%%%%%%%%%%%%%%%%%%%%%%%%%%%%%%%%%%%%%%%%%%%%%%%

\leftline{\bf \large Using Percentages to Describe Change }
There are two basic ways to express the change in the quantity of something.
\begin{enumerate}
\item The \textbf{absolute change} describes the actual increase or decrease from a reference value (starting number) to a new value:
$$ \text{absolute change } = \text{new value } - \text{old value }$$
\item The \textbf{relative change} is a fraction that describes the size of the absolute change in comparison to the reference value:
$$ \text{absolute change } = \frac{ \text{absolute change}}{\text{reference value}} =  \frac{ \text{new value } - \text{old value} }{ \text{old value} } (\times 100\%)$$
\end{enumerate}
Do note that the words \emph{difference} and \emph{change} are interchangeable. 
Sometimes you will see \emph{relative change} and sometimes \emph{relative difference}. Same thing.
\pagebreak

%%%%%%%%%%%%%%%%%%%%%%%%%%%%%%%%%%%%%%%%%%%%%%%%%%
%%%%%%%%%%%%%%%%%%%%%%%%%%%%%%%%%%%%%%%%%%%%%%%%%%
\leftline{\bf \large Practice }
\begin{enumerate}
\item The Dow Jones Industrial Average closed at \$24,849.63 on 12/29/2017 and it closed at \$21,936.01 on 8/31/2017.  What is the absolute change and relative change in the Dow Jones Industrial Average over this time period?
\vspace{0.75in}

\item Suppose that you bought a bike three years ago for \$1275. Today, it is worth only \$550. Describe the absolute and relative change in the bike’s value over this time period.
\vspace{0.75in}

\item The federal minimum wage is \$7.25 per hour. However, Colorado has a higher minimum wage rate of \$8.00 per hour. Compare the federal minimum wage to that in Colorado in both absolute and relative terms. Express your answer as a complete English sentence.
\vspace{0.75in}

\end{enumerate}

%%%%%%%%%%%%%%%%%%%%%%%%%%%%%%%%%%%%%%%%%%%%%%%%%%
%%%%%%%%%%%%%%%%%%%%%%%%%%%%%%%%%%%%%%%%%%%%%%%%%%
\leftline{\bf \large Some More Vocabulary }
\emph{Of} versus \emph{More Than} / \emph{Less Than}
\begin{itemize}
\item If the new or compared value is P\% \emph{more} than the reference value, it is (100 + P)\% of the reference value.
\item If the new or compared value is P\% \emph{less} than the reference value, it is (100 − P)\% of the reference value.
\item[Note:] The quantity ``100\%'' is interchangeable with ``1'' (``one''). \\
\emph{It grew by 25\% = It is 125\% of its original value = It grew by a fourth}. \\
Get comfortable being fluid with these interpretations/vocabulary.
\end{itemize}
\vspace{0.5in}

%%%%%%%%%%%%%%%%%%%%%%%%%%%%%%%%%%%%%%%%%%%%%%%%%%
%%%%%%%%%%%%%%%%%%%%%%%%%%%%%%%%%%%%%%%%%%%%%%%%%%
\leftline{\bf \large Practice }
Express the following as percentages compared to the original value:
\begin{enumerate}
\item It shrank by ten percent. It is now \underline{\hspace{0.5in}}\% of its original size.
\item It is fifteen percent bigger. It is now \underline{\hspace{0.5in}}\% of its original size.
\item I have twice as much as you. I have \underline{\hspace{0.5in}}\% more than you. \\
I have \underline{\hspace{0.5in}}\% of what you have.
\item I have a fourth of what you have.I have \underline{\hspace{0.5in}} less than you. \\
I have \underline{\hspace{0.5in}}\% of what you have.
\end{enumerate}

\pagebreak
%%%%%%%%%%%%%%%%%%%%%%%%%%%%%%%%%%%%%%%%%%%%%%%%%%
%%%%%%%%%%%%%%%%%%%%%%%%%%%%%%%%%%%%%%%%%%%%%%%%%%
\leftline{\bf \large Groupwork }
\begin{enumerate}
\item In 2010, the median income of full-time year-round workers was approximately \$42,800 for men and \$34,700 for women.  These figures do not take into account differences in experience, occupation, education, or overtime hours (\emph{so be careful with possible conclusions}). How many times larger is the full time year round income for men than for women?
\vspace{1in}

\item A store is having a “25\% off” sale.  If the original price of an item is \$34.95, how does the sale price compare to its original price? What is the sale price?
\vspace{1in}

\item (challenge) An item costs you \$8 (ignore tax for now). You know this item was purchased at a 20\% discount. What was the original price of the item? \textit{Show your work, even if you think you have the answer... that's the hardest part.}
\vspace{1in}

\end{enumerate}
%%%%%%%%%%%%%%%%%%%%%%%%%%%%%%%%%%%%%%%%%%%%%%%%%%
%%%%%%%%%%%%%%%%%%%%%%%%%%%%%%%%%%%%%%%%%%%%%%%%%%
\leftline{\bf \large Example }
Do not let ``percentages of percentages'' trip you up. If something goes from 50\% to 75\%, this would correspond to an \textbf{absolute increase} of 25\% and a \textbf{relative increase} of 50\% (since 25 is half of fifty). 

Similarly, if a bank increases its interest rate from 3\% to 4\%, the interest rate increased by (an absolute difference) of 1\% (``one percentage point'' if you prefer).
Now we figure out \emph{what proportion of 3\%} does 1\% represent?
\[
\text{relative change } = \frac{4\% - 3\%}{3\%}\times 100\% = \frac{1}{3}\times 100\% = 33.33\%
\]

If on the other hand the interest rate \emph{decreased} from 4\% to 3\%, this same absolute difference of one percentage point now represents a \emph{25\% decrease}, since 1\% is one-fourth of 4\%. Let's practice writing this process out:
\[
\text{relative change } = \frac{\hspace{0.5in}}{\hspace{0.5in}}\times 100\% = 
\]

\pagebreak
%%%%%%%%%%%%%%%%%%%%%%%%%%%%%%%%%%%%%%%%%%%%%%%%%%
%%%%%%%%%%%%%%%%%%%%%%%%%%%%%%%%%%%%%%%%%%%%%%%%%%
\leftline{\bf \large Abuses of Percentages }
We must be very critical when reading about percentages because they can easily mislead us. Here are some things to watch out for:
\begin{itemize}
\item Beware of shifting reference values.\\
\emph{Ex: a ten percent pay cut followed by a 10 percent pay raise.}\\
Why is this wrong?
\vspace{0.5in}

\item Less than nothing.\\
\emph{Ex: Decrease your caloric intake by 125\% and you are guaranteed to lose weight!}\\
Why is this wrong?
\vspace{0.5in}

\item Do not average percentages.\\
\emph{If 70\% of the boys and 60\% of the girls in a class voted to go to a water park, then 65\% of the students in the class voted to go to the water park.}\\
Why is this wrong?
\vspace{0.5in}
\end{itemize}

%%%%%%%%%%%%%%%%%%%%%%%%%%%%%%%%%%%%%%%%%%%%%%%%%%
%%%%%%%%%%%%%%%%%%%%%%%%%%%%%%%%%%%%%%%%%%%%%%%%%%
\leftline{\bf \large More Groupwork }
\begin{enumerate}
\item You purchase a shirt with a labeled (pre-tax) price of \$21. The local sales tax rate is 6\%. What is your final cost (including tax)?
\vspace{2in}
\item A pair of \$699.95 skis are listed at a clearance price of 50\% off.  What is the sale price for these skis if everything from the store (even sale and clearance items are) 20\% off? 
\emph{How might someone possibly interpret this sale incorrectly?}
\vspace{2in}
\end{enumerate}

\end{document}
