\documentclass[12pt]{article}
\usepackage{epsfig}
\usepackage{amsmath}
\usepackage{amssymb}
\usepackage{graphicx}
\usepackage{float}
\usepackage{epstopdf}

\setlength{\textwidth}{6.2in}
\setlength{\oddsidemargin}{0.3in}
\setlength{\evensidemargin}{0in}
\setlength{\textheight}{8.7in}
\setlength{\voffset}{-.7in}
\setlength{\headsep}{26pt}
\setlength{\parindent}{10pt}
%%%%%%%%%%%%%%%%%%%%%%%%%%%%%%%%%%%%%%%%%%%%%%%%%%%%%%%%%
\begin{document}
\title{\bf Math 1010 \\ Chapter 2A,C}
\date{Jan 22, 2018}
\maketitle

%\vskip .3in


% a few handy macros

\newcommand{\goto}{\rightarrow}
\newcommand{\bigo}{{\mathcal O}}
\newcommand{\half}{\frac{1}{2}}
%\newcommand\implies{\quad\Longrightarrow\quad}
\newcommand\reals{{{\rm l} \kern -.15em {\rm R} }}
\newcommand\complex{{\raisebox{.043ex}{\rule{0.07em}{1.56ex}} \hskip -.35em {\rm C}}}


% macros for matrices/vectors:

% matrix environment for vectors or matrices where elements are centered
\newenvironment{mat}{\left[\begin{array}{ccccccccccccccc}}{\end{array}\right]}
\newcommand\bcm{\begin{mat}}
\newcommand\ecm{\end{mat}}

% matrix environment for vectors or matrices where elements are right justifvied
\newenvironment{rmat}{\left[\begin{array}{rrrrrrrrrrrrr}}{\end{array}\right]}
\newcommand\brm{\begin{rmat}}
\newcommand\erm{\end{rmat}}

% for left brace and a set of choices
\newenvironment{choices}{\left\{ \begin{array}{ll}}{\end{array}\right.}
\newcommand\when{&\text{if~}}
\newcommand\otherwise{&\text{otherwise}}
% sample usage:
%  \delta_{ij} = \begin{choices} 1 \when i=j, \\ 0 \otherwise \end{choices}


% for labeling and referencing equations:
\newcommand{\eql}{\begin{equation}\label}
\newcommand{\eqn}[1]{(\ref{#1})}
% can then do
%  \eql{eqnlabel}
%  ...
%  \end{equation}
% and refer to it as equation \eqn{eqnlabel}.  


% some useful macros for finite difference methods:
\newcommand\unp{U^{n+1}}
\newcommand\unm{U^{n-1}}

% for chemical reactions:
\newcommand{\react}[1]{\stackrel{K_{#1}}{\rightarrow}}
\newcommand{\reactb}[2]{\stackrel{K_{#1}}{~\stackrel{\rightleftharpoons}
   {\scriptstyle K_{#2}}}~}


%
% a few handy macros

\newcommand{\goto}{\rightarrow}
\newcommand{\bigo}{{\mathcal O}}
\newcommand{\half}{\frac{1}{2}}
%\newcommand\implies{\quad\Longrightarrow\quad}
\newcommand\reals{{{\rm l} \kern -.15em {\rm R} }}
\newcommand\complex{{\raisebox{.043ex}{\rule{0.07em}{1.56ex}} \hskip -.35em {\rm C}}}


% macros for matrices/vectors:

% matrix environment for vectors or matrices where elements are centered
\newenvironment{mat}{\left[\begin{array}{ccccccccccccccc}}{\end{array}\right]}
\newcommand\bcm{\begin{mat}}
\newcommand\ecm{\end{mat}}

% matrix environment for vectors or matrices where elements are right justifvied
\newenvironment{rmat}{\left[\begin{array}{rrrrrrrrrrrrr}}{\end{array}\right]}
\newcommand\brm{\begin{rmat}}
\newcommand\erm{\end{rmat}}

% for left brace and a set of choices
\newenvironment{choices}{\left\{ \begin{array}{ll}}{\end{array}\right.}
\newcommand\when{&\text{if~}}
\newcommand\otherwise{&\text{otherwise}}
% sample usage:
%  \delta_{ij} = \begin{choices} 1 \when i=j, \\ 0 \otherwise \end{choices}


% for labeling and referencing equations:
\newcommand{\eql}{\begin{equation}\label}
\newcommand{\eqn}[1]{(\ref{#1})}
% can then do
%  \eql{eqnlabel}
%  ...
%  \end{equation}
% and refer to it as equation \eqn{eqnlabel}.  


% some useful macros for finite difference methods:
\newcommand\unp{U^{n+1}}
\newcommand\unm{U^{n-1}}

% for chemical reactions:
\newcommand{\react}[1]{\stackrel{K_{#1}}{\rightarrow}}
\newcommand{\reactb}[2]{\stackrel{K_{#1}}{~\stackrel{\rightleftharpoons}
   {\scriptstyle K_{#2}}}~}

  % input some useful macros
%\input{exermacros.tex}       % more macros for exercise formatting


% For exercises,
% set enumerate to give parts a, b, c, ...  rather than numbers 1, 2, 3...
%\renewcommand{\theenumi}{\alph{enumi}}
%\renewcommand{\labelenumi}{(\theenumi)}

%%%%%%%%%%%%%%%%%%%%%%%%%%%%%%%%%%%%%%%%%%%%%%%%%%
%%%%%%%%%%%%%%%%%%%%%%%%%%%%%%%%%%%%%%%%%%%%%%%%%%
%%%%%%%%%%%%%%%%%%%%%%%%%%%%%%%%%%%%%%%%%%%%%%%%%%
%%%%%%%%%%%%%%%%%%%%%%%%%%%%%%%%%%%%%%%%%%%%%%%%%%
\leftline{\bf \large Review}
For reference, 
\begin{equation*}
\begin{split}
\text{Recall } \frac{a}{c} \pm \frac{b}{c} &= \frac{a\pm b}{c} \text{ (we need a common denominator for addition/subtraction)}.\\
\text{Also, } \frac{a}{c} \times \frac{b}{d} &= \frac{ab}{cd} \quad \text{ Multiplication is THE core of working with fractions}, \\
&\text{we utilize its flexibility. For example, } a \times \frac{b}{c} = \frac{ab}{c} = b \times \frac{a}{c} .\\
&\text{\emph{Division} is ``multiplication by a reciprocal.'' \emph{Reciprocal = opposite }} \left ( a \text{ and } \frac{1}{a} \right )\\
\text{We have } \frac{a}{b} \div c &= \frac{a}{b} \times \frac{1}{c} \text{ (and, equivalently) } a \div \frac{b}{c} = a \times \frac{c}{b}, \\
& \text{and MOST IMPORTANTLY}, 1 = \frac{a}{a} = \frac{b}{b} = \frac{c}{c} = \dots \text{ etc., (yes, it really is.) }
\end{split}
\end{equation*}
\vspace{.5in}

\leftline{\bf \large Practice}
Simplify / re-write the following fraction using the above information: \\

\begin{equation*}
\frac{\; \frac{a}{b} \;}{\frac{c}{d}}
\end{equation*}
\pagebreak

%%%%%%%%%%%%%%%%%%%%%%%%%%%%%%%%%%%%%%%%%%%%%%%%%%
%%%%%%%%%%%%%%%%%%%%%%%%%%%%%%%%%%%%%%%%%%%%%%%%%%

\leftline{\bf \large Notes on Units (2A)}

\begin{itemize}
\item The {\bf units} of a quantity describe what that quantity \emph{measures} or \emph{counts}
\item {\bf Unit Analysis} is the process of working with units to help solve problems
\end{itemize}

Frequently we encounter scenarios where math problems have no units at all.
However, in our case, we are engaged in \emph{unit analysis}, so units are always going to show up in your final answers and work. 


Referring to the letters above, treat units the same way you would numbers.
There are two key words we need in order to convert ``words into math'' in this section:
\begin{itemize}
\item {\bf per} refers to \emph{division} 
\item {\bf of} refers to \emph{multiplication}. Sometimes instead of ``of'', you will see a hyphen, e.g., watt-hours = watts $\times$ hours.
\item Furthermore, \emph{square} and \emph{cubic} refer to powers, which are \emph{repeated multiplication.}
\end{itemize}

%%%%%%%%%%%%%%%%%%%%%%%%%%%%%%%%%%%%%%%%%%%%%%%%%%
%%%%%%%%%%%%%%%%%%%%%%%%%%%%%%%%%%%%%%%%%%%%%%%%%%
\vspace{0.5in}
\leftline{\bf \large Example 1}
For example, we would use the multiplication rules to commute (move) things around: 
$$
{\bf 1 \text{ ft} \times 2 \text{ ft} }= 1 \times \text{ft} \times 2 \times \text{ft} = 1 \times 2 \times \text{ft} \times \text{ft} = {\bf 2 \text{ ft}^2 }
$$

We just calculated \emph{area}, and \emph{volume} is nothing more than one extra multiplication. 

\emph{Area} and \emph{volume}, along with \emph{length} and \emph{number} constitute our natural notion of size.
\begin{itemize}
\item number = dimensionless.  \emph{(think counting discrete objects)}. ``Zero(th) degree.''
\item length = one dimensional. ``First degree.'' (e.g. $\text{ft}$)
\item area = two dimensional ``Second degree.'' (e.g. $\text{ft}^2$)
\item volume = three dimensional ``Third degree.'' (e.g. $\text{ft}^3$)
\item \emph{also, just be aware that ``degree'' and ``order'' are interchangeable vocabulary}
\end{itemize}

\vspace{1in}
%%%%%%%%%%%%%%%%%%%%%%%%%%%%%%%%%%%%%%%%%%%%%%%%%%
%%%%%%%%%%%%%%%%%%%%%%%%%%%%%%%%%%%%%%%%%%%%%%%%%%

\pagebreak

\leftline{\bf \large Practice (in groups)}
\begin{enumerate}
\item How many minutes are in an hour? How many hours in a day? Use this to compute how many minutes are in a day.
\vspace{1in}

\item How many days are there in a year? Use this and your previous work to compute the number of minutes in a year. 

\vspace{1in}
\item Finally, use these two problems to compute the number of seconds in a year.
\vspace{1in}
\end{enumerate}

\leftline{\bf \large More practice (time permitting)}
How many minutes are in 345 seconds?
\vspace{1in}

%%%%%%%%%%%%%%%%%%%%%%%%%%%%%%%%%%%%%%%%%%%%%%%%%%
%%%%%%%%%%%%%%%%%%%%%%%%%%%%%%%%%%%%%%%%%%%%%%%%%%
\pagebreak
\leftline{\bf \large Example 2}
Similarly, if we have 12 fluid ounces (``fl oz'', or ounce/``oz'' for short when the context is clear) per can of soda, and there are 12 cans per carton, we can compute \emph{how much fluid is in a carton. }

First, let us practice translating these into fractions:
$$
12 \text{ oz per can} = 12 \frac{\text{oz}}{\text{can}} = \frac{12\text{ oz}}{1\text{ can}}
$$
$$
12 \text{ cans per carton} = 12 \frac{\text{can(s)}}{\text{carton}} = \frac{12 \text{ can(s)}}{1 \text{ carton}}
$$
\emph{Notice how we rewrote $12 = \frac{12}{1}$. This comes in handy.} What did we just accomplish? \\

We are establishing the \emph{equivalence} of ``12 ounces'' to ``1 can.''
We are \emph{reformulating} the question into the common structure of fractions (as shown in the Review section of these notes), in order \emph{to make it look like something we know} how to work with mathematically.

You can now essentially treat both these fractions as being the number ``1''. 

This number is special because it is equal to its own reciprocal. 

Which means we can flip the role of numerator and denominator in these quantities however you want. It depends on the question you are trying to ask... ``twelve cans per carton'' is the same as saying ``a can is a twelfth of a carton.''\\
In math terms, I am saying that: 
$$
\frac{12 \text{ can(s)}}{1 \text{ carton}} = \frac{1 \text{ carton}}{12 \text{ can(s)}} = \frac{1}{12} \text{ carton per can, } \text{ AND } 
\frac{12\text{ oz}}{1\text{ can}} = \frac{1\text{ can}}{12\text{ oz}} = \frac{1}{12} \text{ can per oz}
$$
\vspace{0.25in}

Weird? Sure! \\
Useful? Super. Let's see why.
\vspace{0.25in}

\leftline{\bf \large Example 3}
Working off Example 2, we leverage the fact that multiplying one by itself does nothing (maintaining the equivalences, we aren't \emph{changing}, just \emph{converting}), and multiply the two quantities we just worked on expressing together:

$$
{\bf 12 \frac{\text{oz}}{\text{can}} \times 12 \frac{\text{can(s)}}{\text{carton}} } = (12 \times 12) \frac{oz}{carton} \frac{can}{can} = {\bf \frac{144 \text{ oz}}{1 \text{ carton}} }
$$
\vspace{0.25in}

Again, we have established the equivalence of a \emph{carton} to 144 \emph{fluid ounces}.
\textbf{This is something new! }
We did not know this before we did the math. \\
Why might it be useful?
\emph{Why might we care?}\\

Well, we can then divide the price of the carton (let's say it costs \$5.00) by the total amount of fluid inside to get a quantity that we are interested in \emph{price per (fluid) ounce}!\\
First, we establish our common structure:
$$
\text{\$5 per carton} = \frac{\$5}{1\text{ carton}} = \frac{5 \text{ USD}}{1\text{ carton}} = \frac{1\text{ carton}}{5\text{ USD}}
$$
\footnote{Note: \emph{The symbol that denotes the unit for things like the dollar are strange since their units appear \textbf{before} the quantity instead of after.}
Do not let this confuse you, there is nothing different here (multiplication ``commutes,'' remember?) So as strange as it is, $ {\bf \$5 } = \$\times 5 = 5\times \$ = {\bf 5\$} $ is technically completely correct from our perspective.
If you prefer, use the \emph{USD} as a symbol for your unit. }
$$
{\bf \frac{144 \text{ oz}}{1 \text{ carton}} \times \frac{1\text{ carton}}{5\text{ USD}}} = \frac{144}{5} \times \frac{1\times\text{oz}\times \text{ carton}}{1\times\text{carton}\times \text{USD}} 
$$
$$
= \frac{144}{5} \times \frac{\text{oz}}{\text{USD}} \frac{\text{carton}}{\text{carton}} = \frac{144\text{ oz}}{5\text{ USD}} ={\bf \frac{5\text{ USD}}{144\text{ oz}} =   \frac{5}{144} \frac{\text{USD}}{\text{oz}} }
$$
\vspace{0.25in}

We compute that $\frac{5}{144} = 0.0347222.. \approx 0.0347$ and that $\frac{\text{USD}}{\text{oz}}$ means ``dollar per fluid ounce'', so we can conclude that:
\begin{center}
\emph{It costs \$0.0347 per fluid ounce of soda. }
\end{center}
(since there are ``100 pennies per dollar'', we can multiply this by 100 and express our answer in a more useful context:
\begin{center}
\emph{It costs 3.47 cents per fluid ounce of soda. }
\end{center}

And this... is what a lot of math (especially what we will be doing here is all about). 
Utilizing the flexibility of a structure to answer questions we are interested in. 
You might perform that same calculation twice to compare two soda brands in different packages to find out which is cheaper. 

You would do something similar to find ``calories per serving'' or any such familiar quantity.
Now it is time for some practice.

Refer back to these examples to inform your process. 
Ask ``what am I doing'' with each step, and be aware of what quantities you can conclude are equivalent.
\pagebreak

%%%%%%%%%%%%%%%%%%%%%%%%%%%%%%%%%%%%%%%%%%%%%%%%%%
%%%%%%%%%%%%%%%%%%%%%%%%%%%%%%%%%%%%%%%%%%%%%%%%%%
\leftline{\bf \large Practice Problems (Groupwork)}

\begin{enumerate}
\item There are five pieces of gum in a pack. What proportion (\emph{ratio}, \emph{fraction}) of the pack is one piece of gum?
\vspace{0.5in}

\item There are three feet in one yard. 
How many feet are in one square yard?
($1 \text{ yd}^2$). Recall that we have $3\text{ ft} = 1\text{ yd}$. Draw a picture if it helps.
\vspace{1in}

\item What is the total distance traveled when one runs 7 laps around a 400-meter track?
\vspace{0.5in}

\end{enumerate}

%%%%%%%%%%%%%%%%%%%%%%%%%%%%%%%%%%%%%%%%%%%%%%%%%%
%%%%%%%%%%%%%%%%%%%%%%%%%%%%%%%%%%%%%%%%%%%%%%%%%%

\leftline{\bf \large Notes on Problem Solving Guidelines (general) (2C)}
Our textbook breaks these down into four steps:
\begin{enumerate}
\item[\bf Step 1.] Understand the problem.
\item[\bf Step 2.] Devise a strategy.
\item[\bf Step 3.] Carry out your strategy, \emph{revise if necessary.}
\item[\bf Step 4.] Check, interpret and explain your result.
\end{enumerate}

\vspace{0.5in}
\leftline{\bf \large Practice}

Jill and Jack run a 100-meter race. Jill wins by 5 meters \emph{(that is, Jack ran 95 meters by the time Jill crossed the finish line)}. They decide to race again, but this time Jill starts 5 meters behind Jack. Assuming both racers run at the same pace, who will win the second race?

\vspace{2in}


\end{document}
