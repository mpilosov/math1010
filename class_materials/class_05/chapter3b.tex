\documentclass[12pt]{article}
\usepackage{epsfig}
\usepackage{amsmath}
\usepackage{amssymb}
\usepackage{graphicx}
\usepackage{float}
\usepackage{epstopdf}

\setlength{\textwidth}{6.5in}
\setlength{\oddsidemargin}{0in}
\setlength{\evensidemargin}{0in}
\setlength{\textheight}{9.5in}
\setlength{\voffset}{-1in}
\setlength{\headsep}{0.25in}
\setlength{\parindent}{0in}

%%%%%%%%%%%%%%%%%%%%%%%%%%%%%%%%%%%%%%%%%%%%%%%%%%%%%%%%%
\begin{document}
\title{\bf Math 1010\\Chapter 3B}
\date{\vspace{-0.5in}\today}
\maketitle


%\input{./latex/macros.tex}  % input some useful macros
%\input{exermacros.tex}       % more macros for exercise formatting


% For exercises,
% set enumerate to give parts a, b, c, ...  rather than numbers 1, 2, 3...
%\renewcommand{\theenumi}{\alph{enumi}}
%\renewcommand{\labelenumi}{(\theenumi)}


%%%%%%%%%%%%%%%%%%%%%%%%%%%%%%%%%%%%%%%%%%%%%%%%%%
%%%%%%%%%%%%%%%%%%%%%%%%%%%%%%%%%%%%%%%%%%%%%%%%%%

\leftline{\bf \large Notes}
In this section we will work with very large and very small numbers in ways that will help to give them meaning by putting them in perspective.\\

\textbf{Scientific notation} is a format in which a number is expressed as a number between 1 and 10 multiplied by a power of 10.  For example,

$6,700,000,000$ in scientific notation is $6.7 \times 10^9$

$0.000000000000002$ is $2.0 \times 10^{-15}$

This can be really helpful (who wants to write out all those zeros?)
\vspace{0.25in}
    		
To convert a number \textit{from} ordinary notation \textit{to} scientific notation:
\begin{enumerate}
\item Move the decimal point to come after the first nonzero digit.
\item For the power of 10, use the number of places the decimal point moves.
\begin{enumerate}
	\item The power is positive if the decimal point moves to the left.
	\item The power is negative if the decimal point moves to the right.
\end{enumerate}
\end{enumerate}
\vspace{0.25in}


%%%%%%%%%%%%%%%%%%%%%%%%%%%%%%%%%%%%%%%%%%%%%%%%%%
%%%%%%%%%%%%%%%%%%%%%%%%%%%%%%%%%%%%%%%%%%%%%%%%%%
\leftline{\bf \large Practice }
Write each number in scientific notation:
\begin{enumerate}
\item $5769$
\item $8,094,125$
\item $0.0124$
\item $0.00005732$
\end{enumerate}
\vspace{0.25in}


%%%%%%%%%%%%%%%%%%%%%%%%%%%%%%%%%%%%%%%%%%%%%%%%%%
%%%%%%%%%%%%%%%%%%%%%%%%%%%%%%%%%%%%%%%%%%%%%%%%%%
To convert a number \textit{from} scientific notation \textit{to} ordinary notation:
\begin{enumerate}
\item The power of 10 indicates how many places to move the decimal point.
\begin{enumerate}
\item Move it to the right if the power of 10 is positive.
\item Move it to the left if the power of 10 is negative.
\end{enumerate}
\item If moving the decimal place creates any open places, fill them with zeros.
\end{enumerate}

%%%%%%%%%%%%%%%%%%%%%%%%%%%%%%%%%%%%%%%%%%%%%%%%%%
%%%%%%%%%%%%%%%%%%%%%%%%%%%%%%%%%%%%%%%%%%%%%%%%%%
\leftline{\bf \large Practice }
Convert each of the following numbers from scientific to ordinary notation: 
\begin{enumerate}
\item $4 \times 10^3$
\item $7.2 \times 10^8$
\item $9 \times 10^{-4}$
\item $4.378 \times 10^7$
\end{enumerate}
\vspace{0.25in}

%%%%%%%%%%%%%%%%%%%%%%%%%%%%%%%%%%%%%%%%%%%%%%%%%%
%%%%%%%%%%%%%%%%%%%%%%%%%%%%%%%%%%%%%%%%%%%%%%%%%%

\leftline{\bf \large Multiplying with Scientific Notation}
\begin{equation}
\begin{split}
(5 \times 10^3)\times (4\times 10^5) &= (5\times 4) \times (10^3 \times 10^5) \\
&= (20) \times (10^{(3+5)}) \\
&= 2 \times 10 \times (10^8) = 2 \times 10^9
\end{split}
\end{equation}
\vspace{0.5in}

%%%%%%%%%%%%%%%%%%%%%%%%%%%%%%%%%%%%%%%%%%%%%%%%%%
%%%%%%%%%%%%%%%%%%%%%%%%%%%%%%%%%%%%%%%%%%%%%%%%%%

\leftline{\bf \large Practice }
Multiply (without a calculator) and express your answer in scientific notation.
\begin{enumerate}
\item $(7 \times 10^8)\times (2\times 10^{-2})$
\vspace{0.5in}
\item $(5 \times 10^{-3})\times (9\times 10^{-9})$ 
\end{enumerate}
\vspace{0.5in}

\pagebreak
%%%%%%%%%%%%%%%%%%%%%%%%%%%%%%%%%%%%%%%%%%%%%%%%%%
%%%%%%%%%%%%%%%%%%%%%%%%%%%%%%%%%%%%%%%%%%%%%%%%%%
\leftline{\bf \large Multiplying with Scientific Notation}
Division works the same way.
We divide the powers of ten and the other parts of the number separately.  Here is an example:
\begin{equation}
\begin{split}
\frac{(2.1 \times 10^{-3})}{(8.4 \times 10^{-7})} &= \frac{2.1}{8.4} \times \frac{10^{-3}}{10^{7}} \\
\frac{1}{4} \times 10^{-3-7} &= 0.25 \times 10^{-10} \\
&= 2.5 \times 10^{-1} \times 10^{-10} = 2.5 \times 10^{-11}
\end{split}
\end{equation}

%%%%%%%%%%%%%%%%%%%%%%%%%%%%%%%%%%%%%%%%%%%%%%%%%%
%%%%%%%%%%%%%%%%%%%%%%%%%%%%%%%%%%%%%%%%%%%%%%%%%%

\leftline{\bf \large Practice }
Divide (without a calculator) and express your answer in scientific notation.
\begin{enumerate}
\item \Large $\frac{(9 \times 10^{12})}{(3 \times 10^{4})} $
\vspace{0.5in}
\item $\frac{(3.2 \times 10^{5})}{(2\times 10^{-4})}$ 	
\end{enumerate}
\vspace{0.5in}

%%%%%%%%%%%%%%%%%%%%%%%%%%%%%%%%%%%%%%%%%%%%%%%%%%
%%%%%%%%%%%%%%%%%%%%%%%%%%%%%%%%%%%%%%%%%%%%%%%%%%

\leftline{\bf \large Giving Meaning to Numbers}
There is no single recipe for giving meaning to numbers by putting them in perspective, but a few simple techniques can be helpful.  We will look at three techniques: estimation, comparisons, and scaling.\\


\leftline{\bf \large Perspective through Estimation}
An \textbf{order of magnitude} estimate specifies a broad range of values, usually within one or two powers of ten, such as “in the ten thousands” or “in the millions.”  Examples:\\

``The US population is about 313.9 million people.'' = ``The population of the United States is on the order of 300 million people.''
\vspace{0.5in}
\textbf{``The world population is estimated to be about 7,258,000,000 people. ''} = 
\vspace{0.5in}

\textbf{``The number of times you blink your eyelids in a day. \footnote{note: you blink about fifteen times per minute}'' }= 
\vspace{0.5in}

\pagebreak
%%%%%%%%%%%%%%%%%%%%%%%%%%%%%%%%%%%%%%%%%%%%%%%%%%
%%%%%%%%%%%%%%%%%%%%%%%%%%%%%%%%%%%%%%%%%%%%%%%%%%
\leftline{\bf \large Perspective through Comparison}
Making comparisons can give meaning to numbers.
Example: Compare the U.S. population to the world population.
World population is approximately seven billion ($7\times 10^9$). US population is approximately three hundred million ($300\time 10^6$). 
\begin{equation}
\frac{(3 \times 10^{8})}{(7\times 10^{9})} = \frac{3}{7} \times 10^{-1} \approx 0.043
\end{equation} 	
The US population is about 4.3\% of the world's population.

\vspace{0.5in}
\emph{Case Study: What is a Billion Dollars?}
How many people can you employ with \$1 billion per year? The median household income in 2015 was about \$56,000 per year.

\vspace{2in}

A dollar bill is 15.6cm long, 6.6cm wide, and 0.11mm thick (1000 millimeters = 1 meter = 100 centimeters). 
\emph{How tall would a stack of a billion dollars be?}
\end{document}
