\documentclass[12pt]{article}
\usepackage{epsfig}
\usepackage{amsmath}
\usepackage{amssymb}
\usepackage{graphicx}
\usepackage{float}
\usepackage{epstopdf}

\setlength{\textwidth}{6.5in}
\setlength{\oddsidemargin}{0in}
\setlength{\evensidemargin}{0in}
\setlength{\textheight}{9.5in}
\setlength{\voffset}{-1in}
\setlength{\headsep}{0.25in}
\setlength{\parindent}{0in}


% a few handy macros

\newcommand{\goto}{\rightarrow}
\newcommand{\bigo}{{\mathcal O}}
\newcommand{\half}{\frac{1}{2}}
%\newcommand\implies{\quad\Longrightarrow\quad}
\newcommand\reals{{{\rm l} \kern -.15em {\rm R} }}
\newcommand\complex{{\raisebox{.043ex}{\rule{0.07em}{1.56ex}} \hskip -.35em {\rm C}}}


% macros for matrices/vectors:

% matrix environment for vectors or matrices where elements are centered
\newenvironment{mat}{\left[\begin{array}{ccccccccccccccc}}{\end{array}\right]}
\newcommand\bcm{\begin{mat}}
\newcommand\ecm{\end{mat}}

% matrix environment for vectors or matrices where elements are right justifvied
\newenvironment{rmat}{\left[\begin{array}{rrrrrrrrrrrrr}}{\end{array}\right]}
\newcommand\brm{\begin{rmat}}
\newcommand\erm{\end{rmat}}

% for left brace and a set of choices
\newenvironment{choices}{\left\{ \begin{array}{ll}}{\end{array}\right.}
\newcommand\when{&\text{if~}}
\newcommand\otherwise{&\text{otherwise}}
% sample usage:
%  \delta_{ij} = \begin{choices} 1 \when i=j, \\ 0 \otherwise \end{choices}


% for labeling and referencing equations:
\newcommand{\eql}{\begin{equation}\label}
\newcommand{\eqn}[1]{(\ref{#1})}
% can then do
%  \eql{eqnlabel}
%  ...
%  \end{equation}
% and refer to it as equation \eqn{eqnlabel}.  


% some useful macros for finite difference methods:
\newcommand\unp{U^{n+1}}
\newcommand\unm{U^{n-1}}

% for chemical reactions:
\newcommand{\react}[1]{\stackrel{K_{#1}}{\rightarrow}}
\newcommand{\reactb}[2]{\stackrel{K_{#1}}{~\stackrel{\rightleftharpoons}
   {\scriptstyle K_{#2}}}~}


\usepackage{etoolbox}

\newtoggle{sol}
%\toggletrue{sol}

\togglefalse{sol}
\newtheorem{theorem}{Theorem}
\newtheorem{formula}[theorem]{Formula}

%%%%%%%%%%%%%%%%%%%%%%%%%%%%%%%%%%%%%%%%%%%%%%%%%%%%%%%%%
\begin{document}
\title{\bf Math 1010 \\ Chapter 4D\iftoggle{sol}{ Key}{} }
\date{\today}
\maketitle

%%%%%%%%%%%%%%%%%%%%%%%%%%%%%%%%%%%%%%%%%%%%%%%%%%
%%%%%%%%%%%%%%%%%%%%%%%%%%%%%%%%%%%%%%%%%%%%%%%%%%
%%%%%%%%%%%%%%%%%%%%%%%%%%%%%%%%%%%%%%%%%%%%%%%%%%
%%%%%%%%%%%%%%%%%%%%%%%%%%%%%%%%%%%%%%%%%%%%%%%%%%
{\large \bf Loan Basics}
\begin{itemize}
	\item The {\bf principal} is the amount of money owed at a given time.
	\item An {\bf installment loan} (or amortized loan) is a loan that is paid off with equal regular payments (this is the type of loan we will consider).
	\item The {\bf loan term} is the time you have to pay back the loan in full.
\end{itemize}

{\large \bf Mortgages}
\begin{itemize}
	\item A {\bf fixed rate mortgage} is one in which the interest rate is guaranteed not to change over the life of the loan.
	\item An {\bf adjustable rate mortgage} is one where the interest rate changes based on the prevailing rates.
\end{itemize}

\begin{formula}[Loan Payment Formula (regular payments)]
\begin{equation}\label{eq:loan_formula}
	PMT = \frac{P\times \Big(\frac{APR}{n} \Big)}{\Big[1 - \Big(1 + \frac{APR}{n}\Big)^{(-n\times Y)}\Big]}
\end{equation}
where
\begin{itemize}
	\item PMT is the regular payment amount
	\item P is the starting loan principal (amount borrowed)
	\item APR is the annual percentage rate
	\item $n$ is the number of payment periods per year
	\item Y is the loan term in years
\end{itemize}

\end{formula}

\pagebreak

%%%%%%%%%%%%%%%%%%%%%%%%%%%%%%%%%%%%%%%%%%%%%%%%%%
%%%%%%%%%%%%%%%%%%%%%%%%%%%%%%%%%%%%%%%%%%%%%%%%%%
\leftline{\bf \large Exercise 1 : Student Loan}
Suppose you have student loans totaling $\$7,500$ when you graduate from college.  The interest rate is $APR=9\%$, and the loan term is 10 years (you must pay back the entire amount of the loan within 10 years).

\begin{enumerate}
\item What are your monthly payments?

{\it Solution:

We use Eq.~\eqref{eq:loan_formula} to determine the monthly payments,
\begin{eqnarray*}
	PMT &=& \frac{P\times \Big(\frac{APR}{n} \Big)}{\Big[1 - \Big(1 + \frac{APR}{n}\Big)^{(-n\times Y)}\Big]} \\
	&=& \frac{7500\times \Big(\frac{0.09}{12} \Big)}{\Big[1 - \Big(1 + \frac{0.09}{12}\Big)^{(-12\times 10)}\Big]} \\
	&=& \frac{56.25}{1 - 0.40794} \\
	&=& \$95.01	
\end{eqnarray*}

}
\item How much will you pay over the lifetime of the loan?

{\it Solution:
Your monthly payments are $\$95.01$ and you must pay this amount each month for 10 years,
\begin{eqnarray*}
	\frac{10\, years}{1}\frac{12\, months}{1 year}\frac{\$95.01}{1 month} = \$11,401.20
\end{eqnarray*}

}
\item How much will you pay in interest over the total lifetime of the loan?

{\it Solution:
Subtract the original amount borrowed from the total amount of the original loan.
\begin{eqnarray*}
	\$11,401.20 - \$ 7,500 = \$ 3901.20
\end{eqnarray*}

}

\item For the {\bf first three months}, calculate the amount of your payment that goes towards interest, the amount that goes towards your principal, and the new principal after the payment.

{\it Solution:

{\bf Month 1}
The monthly interest rate is $\frac{APR}{12}=\frac{0.09}{12} = 0.0075$.  So the amount you pay in interest for the first month is the monthly interest rate times the CURRENT PRINCIPAL, which in this case is the full loan anount,
\begin{eqnarray*}
	7500 \times 0.0075 = 56.25
\end{eqnarray*}
So you pay $\$56.25$ in interest for the first month.  We already know from part (a) that your total monthly payments are $\$95.01$.  So the amount of that payment that goes toward the principal is,
\begin{eqnarray*}
	95.01 - 56.25 = \$38.76.
\end{eqnarray*}
The new principal is then the principal at the start of the month minus the amount of your payment that went towards the principal,
\begin{eqnarray*}
	P_1 = 7500 - 38.76 = 7461.24
\end{eqnarray*}

{\bf Month 2}
For the second month, the PRINCIPAL HAS CHANGED.  So, the amount of your second payment that goes towards interest is is now the new principal, $P_1$, times the monthly interest rate,
\begin{eqnarray*}
	7461.24 \times 0.0075 = \$55.96,
\end{eqnarray*}
and the amount you pay towards the principal this month is the monthly payment amount minus the amount that you paid towards interest THIS MONTH,
\begin{eqnarray*}
	95.01 - 55.96 = \$39.05
\end{eqnarray*}
The new principal is then the principal at the start of the month, $P_1$, minus the amount of your payment that went towards the principal,
\begin{eqnarray*}
	P_2 = 7461.24 - 39.05 = 7422.19
\end{eqnarray*}

{\bf Month 3}
For the third month, the PRICIPAL HAS CHANGED AGAIN.  The amount of your second payment that goes towards interest is,
\begin{eqnarray*}
	7422.19 \times 0.0075 = \$55.67,
\end{eqnarray*}
and the amount you pay towards the principal this month is the monthly payment amount minus the amount that you paid towards interest THIS MONTH,
\begin{eqnarray*}
	95.01 - 55.67 = \$39.34
\end{eqnarray*}
The new principal is then the principal at the start of the month, $P_2$, minus the amount of your payment that went towards the principal,
\begin{eqnarray*}
	P_2 = 7422.19 - 39.34 = 7382.85
\end{eqnarray*}
}
\end{enumerate}

\pagebreak

%%%%%%%%%%%%%%%%%%%%%%%%%%%%%%%%%%%%%%%%%%%%%%%%%%
%%%%%%%%%%%%%%%%%%%%%%%%%%%%%%%%%%%%%%%%%%%%%%%%%%
\leftline{\bf \large Exercise 2 : Fixed Rate Mortgage}
You need a loan of $\$100,000$ to buy a new house.  The bank offers a choice of a 15-year loan at an APR of $4.5\%$ or a 30-year loan at an APR of $5\%$.  Compare your monthly payments and total loan cost under the two different loan options.

{\it Solution:

{\bf 15-year loan at $\mathbf{4.5\%}$}

We use Eq.~\eqref{eq:loan_formula} to determine the monthly payments,
\begin{eqnarray*}
	PMT &=& \frac{P\times \Big(\frac{APR}{n} \Big)}{\Big[1 - \Big(1 + \frac{APR}{n}\Big)^{(-n\times Y)}\Big]} \\
	&=& \frac{100,000\times \Big(\frac{0.045}{12} \Big)}{\Big[1 - \Big(1 + \frac{0.045}{12}\Big)^{(-12\times 15)}\Big]} \\
	&=& \$764.99
\end{eqnarray*}
Then, multiply the monthly payment by the loan term (in months) to determine the total of all the payments,
\begin{eqnarray*}
	764.99 \times 12 \times 15 = \$137,698.20
\end{eqnarray*}

\vskip 10pt

{\bf 30-year loan at $\mathbf{5\%}$}

We use Eq.~\eqref{eq:loan_formula} to determine the monthly payments,
\begin{eqnarray*}
	PMT &=& \frac{P\times \Big(\frac{APR}{n} \Big)}{\Big[1 - \Big(1 + \frac{APR}{n}\Big)^{(-n\times Y)}\Big]} \\
	&=& \frac{100,000\times \Big(\frac{0.05}{12} \Big)}{\Big[1 - \Big(1 + \frac{0.05}{12}\Big)^{(-12\times 30)}\Big]} \\
	&=& \$536.82
\end{eqnarray*}
Then, multiply the monthly payment by the loan term (in months) to determine the total of all the payments,
\begin{eqnarray*}
	536.82 \times 12 \times 30 = \$193,255.78
\end{eqnarray*}

}

\end{document}