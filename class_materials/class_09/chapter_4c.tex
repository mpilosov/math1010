\documentclass[12pt]{article}
\usepackage{epsfig}
\usepackage{amsmath}
\usepackage{amssymb}
\usepackage{graphicx}
\usepackage{float}
\usepackage{epstopdf}

\setlength{\textwidth}{6.5in}
\setlength{\oddsidemargin}{0in}
\setlength{\evensidemargin}{0in}
\setlength{\textheight}{9.5in}
\setlength{\voffset}{-1in}
\setlength{\headsep}{0.25in}
\setlength{\parindent}{0in}


% a few handy macros

\newcommand{\goto}{\rightarrow}
\newcommand{\bigo}{{\mathcal O}}
\newcommand{\half}{\frac{1}{2}}
%\newcommand\implies{\quad\Longrightarrow\quad}
\newcommand\reals{{{\rm l} \kern -.15em {\rm R} }}
\newcommand\complex{{\raisebox{.043ex}{\rule{0.07em}{1.56ex}} \hskip -.35em {\rm C}}}


% macros for matrices/vectors:

% matrix environment for vectors or matrices where elements are centered
\newenvironment{mat}{\left[\begin{array}{ccccccccccccccc}}{\end{array}\right]}
\newcommand\bcm{\begin{mat}}
\newcommand\ecm{\end{mat}}

% matrix environment for vectors or matrices where elements are right justifvied
\newenvironment{rmat}{\left[\begin{array}{rrrrrrrrrrrrr}}{\end{array}\right]}
\newcommand\brm{\begin{rmat}}
\newcommand\erm{\end{rmat}}

% for left brace and a set of choices
\newenvironment{choices}{\left\{ \begin{array}{ll}}{\end{array}\right.}
\newcommand\when{&\text{if~}}
\newcommand\otherwise{&\text{otherwise}}
% sample usage:
%  \delta_{ij} = \begin{choices} 1 \when i=j, \\ 0 \otherwise \end{choices}


% for labeling and referencing equations:
\newcommand{\eql}{\begin{equation}\label}
\newcommand{\eqn}[1]{(\ref{#1})}
% can then do
%  \eql{eqnlabel}
%  ...
%  \end{equation}
% and refer to it as equation \eqn{eqnlabel}.  


% some useful macros for finite difference methods:
\newcommand\unp{U^{n+1}}
\newcommand\unm{U^{n-1}}

% for chemical reactions:
\newcommand{\react}[1]{\stackrel{K_{#1}}{\rightarrow}}
\newcommand{\reactb}[2]{\stackrel{K_{#1}}{~\stackrel{\rightleftharpoons}
   {\scriptstyle K_{#2}}}~}


\usepackage{etoolbox}

\newtoggle{sol}
%\toggletrue{sol}
\togglefalse{sol}
\newtheorem{theorem}{Theorem}
\newtheorem{formula}[theorem]{Formula}

%%%%%%%%%%%%%%%%%%%%%%%%%%%%%%%%%%%%%%%%%%%%%%%%%%%%%%%%%
\begin{document}
\title{\bf Math 1010 \\ Chapter 4C\iftoggle{sol}{ Key}{} }
\date{\vspace{-0.5in}Feb 14, 2018}
\maketitle


% For exercises,
% set enumerate to give parts a, b, c, ...  rather than numbers 1, 2, 3...
\renewcommand{\theenumi}{\alph{enumi}}
\renewcommand{\labelenumi}{(\theenumi)}
Let's say you decide to save some money by putting a fixed amount away at regular intervals. 
If you put that money into a savings account, it grows with interest. Just like figuring out equal-sized debt payments to pay off debt, there is a formula for figuring out how much you would earn in a savings account! 
Similarly, if you have a goal in mind, you can use a rearranged version of the formula to figure out how much to put away each month.

\begin{formula}[Saving Plan Formula (Regular Payments)]
If you are asked to find the accumulated savings plan balance use:
	\begin{equation}\label{eq:savings_plan_regular}
		A = PMT \times \frac{\Big[\Big(1 + \frac{APR}{n}\Big)^{(n\times Y)} - 1\Big]}{\Big(\frac{APR}{n}\Big)}
	\end{equation}
	OR if you are given a goal of an accumulated balance and asked to find the necessary payment to acheive this goal use:
	\begin{equation}\label{eq:savings_plan_regular_pmt}
		PMT = A \times \frac{\Big(\frac{APR}{n}\Big)}{\Big[\Big(1 + \frac{APR}{n}\Big)^{(n\times Y)} - 1\Big]}
	\end{equation}
	where
	\begin{itemize}
		\item $A$ is the accumulated savings plan balance
		\item $PMT$ is the regular payment (deposit) amount
		\item $APR$ is the Annual Percentage Rate  (as a decimal)
		\item $n$ is the number of payment periods per year
		\item $Y$ is the number of years
	\end{itemize}
\end{formula}

\begin{formula}[Total Return]
	The {\bf total return} is the percentage change in the investment value:
	\begin{equation}\label{eq:total}
		\text{total return} = \frac{\Big(A-P\Big)}{P}\times 100\%
	\end{equation}
	where $A$ is the new value and $P$ is the starting principal.
\end{formula}

\begin{formula}[Annual Return]
	The {\bf annual return} is the annual percentage yield (APY) that would give the same overall growth.
	\begin{equation}\label{eq:annual}
		\text{annual return} = \Big[\Big(\frac{A}{P}\Big)^{(1 / Y)} - 1\Big] \times 100\%
	\end{equation}
	where $A$ is the new value and $P$ is the starting principal.
\end{formula}

%%%%%%%%%%%%%%%%%%%%%%%%%%%%%%%%%%%%%%%%%%%%%%%%%%
%%%%%%%%%%%%%%%%%%%%%%%%%%%%%%%%%%%%%%%%%%%%%%%%%%
\leftline{\bf \large Exercise 1 : Determine Savings Plan Balance}
Suppose you put $\$200$ in a bank account every month that has an APR of $6\%$.

\begin{enumerate}
\item How much money will you have in the account after 6 months?
\iftoggle{sol}{
{\it Solution:
We use Eq.~\eqref{eq:savings_plan_regular} to find the accumulated saving plan balance after 6 months:
\begin{eqnarray*}
	A &=& PMT \times \frac{\Big[\Big(1 + \frac{APR}{n}\Big)^{(n\times Y)} - 1\Big]}{\Big(\frac{APR}{n}\Big)} \\
	&=& \$200 \times \frac{\Big[\Big(1 + \frac{0.06}{12}\Big)^{(12\times 0.5)} - 1\Big]}{\Big(\frac{0.06}{12}\Big)} \\
	&=& \$1215.10
\end{eqnarray*}
}
}{\vspace{2in}}
\item How much money will you have in the account after 10 years?

\iftoggle{sol}{
{\it Solution:
We use Eq.~\eqref{eq:savings_plan_regular} to find the accumulated saving plan balance after 6 months:
\begin{eqnarray*}
	A &=& PMT \times \frac{\Big[\Big(1 + \frac{APR}{n}\Big)^{(n\times Y)} - 1\Big]}{\Big(\frac{APR}{n}\Big)} \\
	&=& \$200 \times \frac{\Big[\Big(1 + \frac{0.06}{12}\Big)^{(12\times 10)} - 1\Big]}{\Big(\frac{0.06}{12}\Big)} \\
	&=& \$32,775.87
\end{eqnarray*}
}
}{\vspace{1in}}

\end{enumerate}

%%%%%%%%%%%%%%%%%%%%%%%%%%%%%%%%%%%%%%%%%%%%%%%%%%
%%%%%%%%%%%%%%%%%%%%%%%%%%%%%%%%%%%%%%%%%%%%%%%%%%
\pagebreak
\leftline{\bf \large Exercise 2 : Determine Savings Plan Payments}
Suppose you want to put a downpayment on a house in 6 years and expect the downpayment to cost $\$35,000$.  Your bank offers a plan with a guaranteed APR of $4.5\%$ if you make regular monthly deposits.

\begin{enumerate}
\item How much should you deposit each month to end up with $\$35,000$ in 6 years?

\iftoggle{sol}{
{\it Solution:
We use Eq.~\eqref{eq:savings_plan_regular_pmt} to find the necessary payment to acheive the investment goal of $\$35,000$:
\begin{eqnarray*}
	PMT &=& A \times \frac{\Big(\frac{APR}{n}\Big)}{\Big[\Big(1 + \frac{APR}{n}\Big)^{(n\times Y)} - 1\Big]} \\
	&=& \$35000 \times \frac{\Big(\frac{0.045}{12}\Big)}{\Big[\Big(1 + \frac{0.045}{12}\Big)^{(12\times 6)} - 1\Big]} \\
	&=& \$424.34 \text{ per month}
\end{eqnarray*}
}
}{\vspace{1.5in}}

\item How much should you deposit each month to end up with $\$35,000$ in 6 years if the APR was instead 5.5\%?

\iftoggle{sol}{
{\it Solution:
We use Eq.~\eqref{eq:savings_plan_regular_pmt} to find the necessary payment to acheive the investment goal of $\$35,000$:
\begin{eqnarray*}
	PMT &=& A \times \frac{\Big(\frac{APR}{n}\Big)}{\Big[\Big(1 + \frac{APR}{n}\Big)^{(n\times Y)} - 1\Big]} \\
	&=& \$35000 \times \frac{\Big(\frac{0.055}{12}\Big)}{\Big[\Big(1 + \frac{0.055}{12}\Big)^{(12\times 6)} - 1\Big]} \\
	&=& \$411.41 \text{ per month}
\end{eqnarray*}
}
}{\vspace{1.5in}}

\item Suppose you are saving for a 1 million dollar home and you need to put down a $10\%$ deposit in 6 years with the 4.5\% APR plan above.  How much should you deposit each month to end up with $\$100,000$ ($10\%$ of 1 million) in 6 years?

\iftoggle{sol}{
{\it Solution:

We use Eq.~\eqref{eq:savings_plan_regular_pmt} to find the necessary payment to acheive the investment goal of $\$100,000$:
\begin{eqnarray*}
	PMT &=& A \times \frac{\Big(\frac{APR}{n}\Big)}{\Big[\Big(1 + \frac{APR}{n}\Big)^{(n\times Y)} - 1\Big]} \\
	&=& \$100000 \times \frac{\Big(\frac{0.045}{12}\Big)}{\Big[\Big(1 + \frac{0.045}{12}\Big)^{(12\times 6)} - 1\Big]} \\
	&=& \$1212.40
\end{eqnarray*}
}
}{\vspace{1.5in}}
\end{enumerate}

%%%%%%%%%%%%%%%%%%%%%%%%%%%%%%%%%%%%%%%%%%%%%%%%%%
%%%%%%%%%%%%%%%%%%%%%%%%%%%%%%%%%%%%%%%%%%%%%%%%%%
\pagebreak
\leftline{\bf \large Exercise 3 : Determine Total and Annual Return}
You invest $\$3000$ is a mutual fund.  Over 4 years, you investment grows in value to $\$8400$.

\begin{enumerate}
\item What is the total return on this investment?
\iftoggle{sol}{
{\it Solution:

We use Eq.~\eqref{eq:total} with $P=\$3000$ and $A=\$8400$ to find the total return:
\begin{eqnarray*}
	\text{total return} &=& \frac{\Big(A-P\Big)}{P}\times 100\% \\
	&=& \frac{\$\Big(8400-3000\Big)}{\$3000}\times 100\% \\
	&=& 1.8 \times 100\% \\
	&=& 180 \%
\end{eqnarray*}

}

}{\vspace{2in}}
\item What is the annual return on this investment?

\iftoggle{sol}{
{\it Solution:

We use Eq.~\eqref{eq:annual} with $P=3000$ and $A=8400$ to find the total return:
\begin{eqnarray*}
	\text{annual return} &=& \Big[\Big(\frac{A}{P}\Big)^{(1 / Y)} - 1\Big] \times 100\% \\
	&=& \Big[\Big(\frac{8400}{3000}\Big)^{(1 / 4)} - 1\Big] \times 100\% \\
	&=& 0.29 \times 100\% \\
	&=& 29 \%
\end{eqnarray*}

}
}{\vspace{2in}}
\end{enumerate}



\end{document}
