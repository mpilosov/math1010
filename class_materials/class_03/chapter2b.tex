\documentclass[12pt]{article}
\usepackage{epsfig}
\usepackage{amsmath}
\usepackage{amssymb}
\usepackage{graphicx}
\usepackage{float}
\usepackage{epstopdf}

\setlength{\textwidth}{6.2in}
\setlength{\oddsidemargin}{0.3in}
\setlength{\evensidemargin}{0in}
\setlength{\textheight}{8.7in}
\setlength{\voffset}{-.7in}
\setlength{\headsep}{26pt}
\setlength{\parindent}{10pt}
%%%%%%%%%%%%%%%%%%%%%%%%%%%%%%%%%%%%%%%%%%%%%%%%%%%%%%%%%
\begin{document}
\title{\bf Math 1010 \\ Chapter 2B}
\date{Jan 24, 2018}
\maketitle

%\vskip .3in

\input{../macros}
%\input{./latex/macros.tex}  % input some useful macros
%\input{exermacros.tex}       % more macros for exercise formatting


% For exercises,
% set enumerate to give parts a, b, c, ...  rather than numbers 1, 2, 3...
%\renewcommand{\theenumi}{\alph{enumi}}
%\renewcommand{\labelenumi}{(\theenumi)}


%%%%%%%%%%%%%%%%%%%%%%%%%%%%%%%%%%%%%%%%%%%%%%%%%%
%%%%%%%%%%%%%%%%%%%%%%%%%%%%%%%%%%%%%%%%%%%%%%%%%%

\leftline{\bf \large Notes}
Our book outlines the following process for approaching problems we might find unfamiliar: \\
\begin{enumerate}

\item Step 1. Identify the units involved in the problem and the units that you expect for the answer.

\item Step 2. Use the given units and the expected answer units to help you find a strategy for solving the problem.
Perform all operations (such as multiplication or division) on both the numbers and their associated units.  \emph{Remember:}
\begin{enumerate}
\item You cannot add or subtract numbers with different units, but you can combine different units through multiplication, division, or raising to powers.
\item It is easier to keep track of units if you \textbf{always replace division with multiplication by the reciprocal.} 
For example, instead of dividing by  $\frac{60\text{ sec}}{\text{ min}}$, multiply by $\frac{1\text{ min}}{60\text{ sec}}$. 
\emph{This might take some getting used to at first but hopefully you will never use the division symbol ever again, considering it is just a way to avoid writing a fraction, and is more subject to mis-interpretation than clearly showing what is in the numerator and what is in the denominator.}
\end{enumerate}

\item Step 3. When you complete your calculations, make sure that your answer has the units you expected.  If it does not, then you have done something incorrect.  Check to see of the answer makes sense (ask, \emph{does this seem reasonable}, use your intuition).
\end{enumerate}
\vspace{2in}

%%%%%%%%%%%%%%%%%%%%%%%%%%%%%%%%%%%%%%%%%%%%%%%%%%
%%%%%%%%%%%%%%%%%%%%%%%%%%%%%%%%%%%%%%%%%%%%%%%%%%

\leftline{\bf \large Example 1}
You are buying 30 acres of farm land at \$12,000 per acre. What is the total cost of the land?

\leftline{\bf Solution}
\begin{enumerate}
\item Step 1. The units involved in the problem are \underline{\hspace{1in}} and \underline{\hspace{1in}}.
The units in the answer should be \underline{\hspace{1in}}.

\item Step 2. We can end up with an answer in dollars by multiplying the acreage by the cost per acre:
\vspace{1in}


Step 3. Has the answer come out with the units we expect?  Does the answer make sense?
\vspace{0.5in}
\end{enumerate}

\leftline{\bf \large Example 2}
You are planning to make pesto and need to buy basil.  At the grocery store, you can buy small containers of basil priced at 
\$2.99 for each 2/3-ounce container.  At the farmer’s market, you can buy basil in bunches for \$12 per pound.  Which is the better deal?

\leftline{\bf Solution}
\begin{enumerate}
\item Step 1.  Think \emph{What does it mean to be the ``better deal?'' What kind of problem might we solve to figure this out?} 
The units involved in the problem are \underline{\hspace{1in}} and \underline{\hspace{1in}} and \underline{\hspace{1in}}.
	To compare the prices, we need them both in the same units.  We have a choice here.  They could both be in the units \underline{\hspace{1in}} or in \underline{\hspace{1in}}.

\item Step 2. Convert the small container price to a \emph{price per pound} to answer the question. 

\vspace{1in}

Step 3. Has the answer come out with the units we expect?  Does the answer make sense? \emph{How else might we have done the problem?}

\end{enumerate}
\vspace{2in}
\pagebreak

\leftline{\bf \large Example 3}
Your destination is 90 miles away, and your fuel gauge shows that your gas tank is one-quarter full. Your tank holds 12 gallons of gas, and your car averages about 25 miles per gallon. Do you need to stop for gas?

\leftline{\bf Solution}
\begin{enumerate}
\item Step 1. The units involved in the problem are \underline{\hspace{1in}}. The units in the answer should be \underline{\hspace{1in}}.

\item Step 2.
\vspace{1in}

\item Step 3. Has the answer come out with the units we expect?  Does the answer make sense?
\vspace{1in}
\end{enumerate}
\vspace{2in}



\end{document}
