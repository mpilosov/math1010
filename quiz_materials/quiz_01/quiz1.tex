% TEMPLATE FOR STANDARD QUIZ
\documentclass[10.5pt]{article}
\usepackage{amssymb,amsmath,amsthm,mathrsfs}
\usepackage{graphicx}
\oddsidemargin=0in
\evensidemargin=0in
\textwidth=6.3in
\topmargin=-0.5in
\textheight=9in

% packages for fancy fonts, symbols, thm/proof environments, etc
\usepackage{amsmath,amssymb,amsthm}
\usepackage{setspace}
\usepackage{nicefrac}
\usepackage{enumitem}
\usepackage{calc}
\usepackage{soul}



%\usepackage[sc]{mathpazo}
%\linespread{1.05}         % Palatino needs more leading (space between lines)
%\usepackage[T1]{fontenc}
\usepackage{tgpagella}
\usepackage[T1]{fontenc}
\usepackage[multiple]{footmisc}

\parindent=0in
\pagestyle{empty}
\newcommand{\Fbox}[1]{\fbox{\strut#1}}
\setlength{\fboxsep}{1pt}% Just for this example
\setlength{\parindent}{0pt}% Just for this example

\input{../testpoints}

\newcounter{choice}
\renewcommand\thechoice{\Alph{choice}}
\newcommand\choicelabel{\thechoice.}

\newenvironment{choices}%
  {\list{\choicelabel}%
     {\usecounter{choice}\def\makelabel##1{\hss\llap{##1}}%
       \settowidth{\leftmargin}{W.\hskip\labelsep\hskip 2.5em}%
       \def\choice{%
         \item
       } % choice
       \labelwidth\leftmargin\advance\labelwidth-\labelsep
       \topsep=0pt
       \partopsep=0pt
     }%
  }%
  {\endlist}

\newenvironment{oneparchoices}%
  {%
    \setcounter{choice}{0}%
    \def\choice{%
      \refstepcounter{choice}%
      \ifnum\value{choice}>1\relax
        \penalty -50\hskip 1em plus 1em\relax
      \fi
      \choicelabel
      \nobreak\enskip
    }% choice
    % If we're continuing the paragraph containing the question,
    % then leave a bit of space before the first choice:
    \ifvmode\else\enskip\fi
    \ignorespaces
  }%
  {}

\begin{document}


%%%(change to appropriate class and semester)
MATH 1010-004 Spring 2018

%%%(change to appropriate quiz type and date)
Quiz 1 (\today) \hspace{1.9in} \qquad {Name:} {\underline {\hspace{2.5in}}}
\vspace{2pc}

%%%(modify rules, time, points as appropriate)
You will have 20 minutes to complete the following quiz.   No books or notes allowed.  You are allowed to use calculators.
\vspace{1pc}

\begin{problem}{4} Use a set notation to write the members of the following sets.
\begin{description}
\item{(a)} The vowels of the English alphabet
\vspace{1in}
\item{(b)} The even numbers between, but not including, 12 and 100
\vspace{1in}
\end{description}
\end{problem}

\begin{problem}{7} Qdoba sells chicken quesadillas with the choice of sour cream and guacamole. One day they sold 190 chicken quesadillas; 120 had sour cream, 142 had guacamole, 84 had both sour cream and guacamole.
\begin{description}
\item{(a)} Draw a Venn diagram to illustrate the situation.  Be sure to label all {\bf four} parts of the Venn diagram clearly.
\vspace{2.4in}
\item{(b)} What is the number of chicken quesadillas sold with guacamole only? \vspace{0.18 in} \\
 Answer here : \underline{\hspace{1in}} \\
\item{(c)} What is the number of chicken quesadillas sold without both guacamole and sour cream? \vspace{0.2 in} \\
Answer here : \underline{\hspace{1in}}
\end{description}
\end{problem}

\newpage
\begin{problem}{5} 
Draw a Venn Diagram to represent the relationship between the following parts of sets: 
Nurses and Skydivers. \emph{Label all areas for full credit.}
\vspace{2.5in}
\end{problem}

\begin{problem}{4} Use the Venn diagram to answer the following questions. 
\begin{figure}[!ht]
  \centering
    \includegraphics[width=1.0\textwidth]{Quiz1_p1}
\end{figure}
\begin{description}
\item{(a)} How many people at the conference are employed men without a college degree? 
Answer: \underline{\hspace{0.5in}} \\
\item{(b)} How many people at the conference are unemployed women? 
Answer: \underline{\hspace{0.5in}} \\
\item{(c)} How many people at the conference are unemployed men without a college degree? 
Answer: \underline{\hspace{0.5in}} \\
\item{(d)} How many people are at the conference? 
Answer: \underline{\hspace{0.5in}} \\
\end{description}
\vspace{3.9in}
\end{problem}


\showpoints
\end{document}






